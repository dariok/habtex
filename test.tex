\documentclass[open=right, index=totoc, paper=160mm:234mm, chapterprefix=true,
	fontsize=9pt, DIV=10, BCOR=12mm, headwidth=112.5mm:0mm:0mm,
	headsepline=:112.5mm]{scrbook}
%	fontsize=9pt, DIV=9, BCOR=12mm, headwidth=textwithmarginpar, footwidth=textwithmarginpar]{scrbook}
\usepackage{scrtime}
\KOMAoption{headings}{small,optiontohead}%
\usepackage{scrlayer-scrpage}
\usepackage{xifthen}
\usepackage{fontspec}
\usepackage{setspace}
\usepackage{multirow}
\usepackage{chngcntr}
\usepackage{refcount}
\usepackage{geometry}
\usepackage{multicol}
\usepackage{pdflscape}
\usepackage{afterpage}
\usepackage{longtable}
\usepackage{tabu}
%\usepackage{xr}
\usepackage{expl3}
%\geometry{inner=23mm, outer=32mm, top=25mm, bottom=30mm, headsep=5mm, marginparsep=2.5mm,marginparwidth=26mm}%,
% showframe}% normales Layout
\geometry{left=25mm, right=22.5mm, top=22.5mm, bottom=30mm, headsep=2.5mm, marginparsep=0mm, marginparwidth=0mm,
	asymmetric}%, showframe}%%%%%%%%%%%%%%%%%%%%%%%

%\usepackage[a4]{crop}

% Schriftarten einstellen
% Hauptschriftart ist Junicode mit klassischen TeX-Ersetzungen und Mediävalziffern
\setmainfont[Ligatures=TeX]{Junicode}
%\KOMAoptions{fontsize=9pt}%%%

% Kürzungsstriche
\newcommand{\abk}[2][3]{
	\nolinebreak\hspace{.1em}\nolinebreak\rule[#1pt]{0.3pt}{#1pt}\nolinebreak\hspace{.1em}\nolinebreak%
	#2%
	\nolinebreak\hspace{.1em}\nolinebreak\rule[#1pt]{0.3pt}{#1pt}\nolinebreak\hspace{.1em}\nolinebreak%
}

% Bearbeiter in der Introduction %
\newcommand{\bearb}[1]{%
	%\makebox[\textwidth]{\small{#1}}%
	\begin{center}\small #1\end{center}%
}

% Kopfzeile umdefinieren und Fußzeile ausschalten
\pagestyle{scrheadings}
\clearscrheadfoot
\cehead{\textnormal{\small\leftmark}}
\lehead{\pagemark}
\cohead{\textnormal{\small\rightmark}}
\rohead{\pagemark}

\setlength{\parindent}{4mm}
\setlength{\parskip}{0pt plus 0.5pt}
\setstretch{1.05}

%%% FUSSNOTEN %%%%%%%%%%%%%%%%%%%%%%%%%%%%%%%%%%%
% Vorbereiten und Setzen der Fußnotenapparate	%
% und Zeilenzählung								%
% benötigt: eledmac								%
%%%%%%%%%%%%%%%%%%%%%%%%%%%%%%%%%%%%%%%%%%%%%%%%%
\renewcommand{\footnoterule}{%
	\hrule width 112.5mm height 0.4pt
}
%% Fußnotenlinie in voller Länge

% Fußnoten dürfen eigentlich immer umbrechen
\interfootnotelinepenalty=0

\clubpenalty=0
\widowpenalty=0
\displaywidowpenalty=0

\usepackage[nocritical,noeledsec,noend,series={A,B}]{reledmac}
\usepackage{reledpar}
\linenummargin{left}%Marginalienlayout
\lineation{page}
\setlength{\linenumsep}{2.5mm}
\notefontsizeX[A]{\small}
\notefontsizeX[B]{\small}
\renewcommand{\numlabfont}{\normalfont\scriptsize}%

%%% Fußnote A
% alle in einem Absatz, fette Markierung, fester Abstand, Zählung a-z, aa-az, \ldots
\arrangementX[A]{paragraph}
\counterwithin*{footnoteA}{section}

\usepackage{intcalc}
\newcounter{myPFN}
\newcounter{myAFN}
\newcommand{\aalph}[1]{%
	%\textbf{#1.}%
	\IfInteger{#1}{%
		\setcounter{myPFN}{\intcalcDiv{\intcalcAdd{#1}{-1}}{26}}%
		\setcounter{myAFN}{\intcalcAdd{1}{\intcalcMod{\intcalcAdd{#1}{-1}}{26}}}%
	}{%
		\setcounter{myPFN}{\intcalcDiv{\intcalcAdd{\getrefnumber{#1}}{-1}}{26}}%
		\setcounter{myAFN}{\intcalcAdd{1}{\intcalcMod{\intcalcAdd{\getrefnumber{#1}}{-1}}{26}}}%
	}%
	%\arabic{myPFN}-\arabic{myAFN}%
	%\alph{myPFN}.\alph{myAFN}%
	\alph{myPFN}\alph{myAFN}%
}

\renewcommand*{\bodyfootmarkA}{%
	\textsuperscript{\aalph{\thefootnoteA}}%
}

% neu 2016-10-06 DK
% Nutzung von marginFootMark rückgängig gemacht, da Probleme in Listen auftauchen; 2016-12-23 DK
% TODO: Prüfung, ob label existiert – irgendwie anders machen, denn Buchstaben in margine gibt Probleme
\newcommand{\habBodyFootmarkA}[1]{%
	\textsuperscript{\aalph{#1}}%
}
\newcommand{\habMarginFootmarkA}[1]{%
	{\fontsize{5}{5}\selectfont\raisebox{2pt}[0pt]{\aalph{#1}}}%
}

% neu 2016-10-06 DK
\newcommand{\habCritNoteRef}[1]{%
	S. \pageref{#1}, Anm. \aalph{#1}%
}

% Fußnotenmarke in der Fußnote ausschalten, wird für a-a anderweitig gesetzt.
\makeatletter
	\renewcommand*{\footfootmarkA}{}%
\makeatother

% Command \footnotetextA erzeugt den Text der Fußnote und die Markierung: wenn optionaler Param = 1, dann a-a
\makeatletter
	\newcommand{\footnotetextA}[2][2]{%
		\footnoteA{%
			%\textbf{%
				\expandafter\ifnum#1=1\aalph{\@nameuse{@thefnmarkA}}--\aalph{\@nameuse{@thefnmarkA}}%
				\else%
					\ifnum#1>1\aalph{\@nameuse{@thefnmarkA}}\fi%
				\fi%
			%}
			)~#2%
		}%
	}%
\makeatother%

% Zitatbereiche
% schaut nach vorne, um keine doppelten Punkte zu setzen; 01/2017 DK
% ggf. Seitenzahl nicht ausgeben; 2017-03-03 DK
\ExplSyntaxOn
	\newcommand{\lineref}[2][0]{%
		\ifnum#1=1%% Paramter = 1 gesetzt, also Seite unterdrücken
			\relax%
		\else%
			S. \xpageref{#2},\ %
		\fi%
		Z. \edlineref{#2}%
		\ifthenelse{\xpageref{#2}=\xpageref{#2e}}{%
			\ifthenelse{\xlineref{#2}=\xlineref{#2e}}%
			{}%
			{\ifthenelse{\intcalcAdd{\xlineref{#2}}{1}=\xlineref{#2e}}%
					% wenn ein Punkt folgt, keinen ausgeben. 2017-01-28 DK %
					{f%
						\peek_charcode:NTF.{\relax}{.}%
					}%
					{\hspace{0.25pt}–\hspace{0.25pt}\edlineref{#2e}}%
			}%
		}{%
			\hspace{0.25pt}–\hspace{0.25pt}S. \xpageref{#2e},\ Z. \edlineref{#2e}%
		}%
	}%
\ExplSyntaxOff

\makeatletter
	\newcounter{mySelfRef}%
	\newcounter{myChap}%
	\setcounter{mySelfRef}{1}
	\newcommand{\noteref}[2][0]{%
	% prüfen, ob das label überhaupt schon definiert ist, sonst gibt es Probleme; 2016-12-14 DK
	% ggf. die Seite nicht mit ausgeben. 2017-02-23 DK %
	\@ifundefined{r@#2}{000}{%
			\label{e\themyChap noteref\themySelfRef}%
			\ifstr{\pageref{#2}}{\pageref{e\themyChap noteref\themySelfRef}}%
				{Anm.~\ref{#2}}%
				{\ifnum#1=1% Paramter = 1 gesetzt, also Seite unterdrücken
					{Anm.~\ref{#2}}%
				\else%
					{S. \pageref{#2} Anm. \ref{#2}}%
				\fi}%
			\stepcounter{mySelfRef}%
		}%
	}
\makeatother

\ExplSyntaxOn
\makeatletter
	\newcommand{\habPageref}[1]{%
		S. \pageref{#1}%
		% Angepaßt auf ifthenelse, nur \ifnum machte Ärger; 2017-01-18 DK
		\ifthenelse{\getpagerefnumber{#1}<\getpagerefnumber{#1e}}
		{%
			\ifthenelse{\intcalcAdd{\pageref{#1}}{1}=\pageref{#1e}}%
					% wenn ein Punkt folgt, keinen ausgeben. 2017-02-05 wie bei lineref DK %
					{f%
						\peek_charcode:NTF.{\relax}{.}%
					}%
					{–\pageref{#1e}}%
		}{}%
		%\fi%
	}
\makeatother
\ExplSyntaxOff
	
%%% Fußnote B %%%%%%%
% jeweils eigener Absatz, Marke linksbündig max. 7mm, Einzug 7mm
\hangindentX[B]{7mm}%{1.7em}
\makeatletter
	\renewcommand*{\footfootmarkB}{%
		\makebox[7mm][l]{\@nameuse{@thefnmarkB}.\ }%
	}
\makeatother

\counterwithin*{footnoteB}{section}

\newcommand{\bodyrefB}[1]{%
	\textsuperscript{\normalfont{\ref{#1}}}%
}

\newenvironment{marginFoot}{%
	\renewcommand*{\bodyfootmarkB}{}%
	\renewcommand*{\bodyfootmarkA}{}%
	\renewcommand*{\multfootsep}{}%
}{}

\renewcommand*{\multfootsep}{\textsuperscript{,}}

%%%%%%%% ENDE FUSSNOTEN %%%%%%%%%%%

% Indices vorbereiten
\usepackage[split]{splitidx}
\makeindex
\newindex[Personen]{per}
\newindex[Orte]{pla}
\newindex[Bibelstellen]{bib}
%\newindex[Zitate]{cit}
%\newindex[Sachen]{term}

%\usepackage[indentunit=2mm]{idxlayout}%
\usepackage{polyglossia}
\setdefaultlanguage{german}
%\setotherlanguage[variant=medieval]{latin}
\setotherlanguage[variant=modern]{latin}
\setotherlanguage[variant=ancient]{greek}
\setotherlanguage{czech}
\newfontfamily\greekfont{FreeSerif}
\setotherlanguage{hebrew}
\newfontfamily\hebrewfont{FreeSerif}

% Nach reledmac laden?
\usepackage[protrusion=true]{microtype}

\newcommand{\page}[1]{[#1]}

% erweitert analog \reditem; 2016-11-08 DK %
\newcommand{\editem}[2]{%
	\ifnumbering%
		\pstart%
	\else%
		\par%
	\fi%
	\vspace{0.25\baselineskip plus 0.1pt minus 0.1pt}%
	\hangindent=7mm\hangafter=1%
	\noindent\makebox[7mm][l]{#1}#2%
	\ifnumbering\pend\fi%
}
\newcommand{\leditem}[3][7]{%
	\par\hangindent=#1mm\hangafter=1%
	\noindent\makebox[#1mm][l]{#2}#3%
	\vspace{0.25\baselineskip}%
}
% TODO: items vereinheitlichen!
% erweitert, um sowohl in numerierten als auch unnumerierten Abschnitten zu funktionieren; 2016-06-06 DK %
\newcommand{\reditem}[2]{%
	\ifnumbering%
		\pstart%
	\else%
		\par%
	\fi%
	% ein wenig Luft bei den Abständen erlauben für besseren Umbruch; 2017-01-29 DK %
	\vspace{0.1\baselineskip plus 0.1\baselineskip minus 1pt}
	\hangindent=7mm\hangafter=1%
	\noindent #1\hfill\newline#2%
	\ifnumbering\pend\fi%
}
\newcommand{\beditem}[2]{%
	\par\vspace{0.25\baselineskip plus 0.05\baselineskip minus 0.05\baselineskip}%
	\hangindent=7mm\hangafter=1%
	\noindent\textbf{#1} --- #2% 
}
\newlength{\habItemLen}
\newcommand{\flexitem}[3][7mm]{%
	\settowidth{\habItemLen}{#2~}%
	% neu 2017-02-23 DK
	\ifnumbering%
		\pstart%
	\else%
		\par%
	\fi%
	\vspace{0.25\baselineskip}\hangindent=7mm\hangafter=1\noindent%
	\ifdim\habItemLen < #1%
		\makebox[#1][l]{#2}#3%
	\else%
		#2 #3%
	\fi%
	% neu 2017-02-23 DK
	\ifnumbering\pend\fi%
}

% Kompatibilität mit numerierten Abschnitten herstellen; 2016-09-20 DK
\newcommand{\habquote}[1]{%
	\ifnumbering\pstart%
		\else\par
	\fi%
	\vspace{0.5\baselineskip}%
	\hangindent=7mm\hangafter=0%
	{\noindent #1}%
	\vspace{0.5\baselineskip}%
	\ifnumbering\pend%
		\else\relax%\par%
	\fi%
	%\par\vspace{0.25\baselineskip}\noindent%
}

% Neu 2017-02-12
\newenvironment{habQuote}{%
	\ifnumbering\pstart%
		\else\vspace{0.5\baselineskip plus 0.1\baselineskip minus 0.05\baselineskip}\par
	\fi%
	\hangindent=7mm\hangafter=0%
	\noindent%
}{%
	\ifnumbering\pend%
		\else\vspace{0.5\baselineskip plus 0.1\baselineskip minus 0.05\baselineskip}\par%
	\fi%
}
% habMotto für zentrierten Text ohne Abstände innerhalb eines Absatzes und unter einer Zeilenlänge
\newcommand{\habMotto}[1]{%
	\pend%
	\pstart\noindent%
		\makebox[\textwidth][c]{#1}%
	\pend%
	\pstart\noindent%
}

% Marginalie
\setlength{\ledlsnotesep}{2.5mm}
\setlength{\ledrsnotesep}{2.5mm}
\setlength{\ledrsnotewidth}{31mm} % Marginalien-Layout
% FN gehen in den Marginalienraum ueber
\widthX{112.5mm}
\rightnoteupfalse
\leftnoteupfalse
\renewcommand*{\ledlsnotefontsetup}{\raggedleft\scriptsize}
\renewcommand*{\ledrsnotefontsetup}{\raggedright\scriptsize}
\sidenotemargin{right}% Marginalienlayout
\newcommand{\marginalie}[1]{%
	\ledsidenote{\fontsize{6.5}{7.75}\selectfont #1}%
}

% Prüft jetzt, ob Int übergeben wurde (direkt ausgeben), sonst als Referenz behandeln; 2016-12-10 DK
\makeatletter
\newcommand{\marginFootMark}[1]{%
	{\fontsize{5}{5}\selectfont\raisebox{2pt}[0pt]{%
		% nochmal umgeschrieben, testet noch mehr; 2016-12-20 DK
		% Prüfen, ob es ein Befehl ist
		\@ifundefined{#1}%
			% Befehl ist nicht definiert
			{% ist es ein Label?
				\@ifundefined{r@#1}%
					% kein Label, also Text ausgeben. Kommt das vor?
					{#1}%
					% wenn Label, dann auflösen
					{\ref{#1}}%
			}%
			% Befehl ist definiert (vmtl. \aalph), also ausführen
			{#1}%
	}}%
}
\makeatother

% Für Opener
% Zentrierung entfernt; Ausrichtung muß im XML passieren; 2016-06-06 DK %
\newcommand{\habOpener}[1]{%
	\pstart\noindent #1%
	\pend\vspace{0.5\baselineskip}%
}

% Einstellungen für Überschriften
% KOMA-Überschriften funktionieren nicht mit Fußnoten; hier müßte sonst eine Lösung wie bei Marginalien her
% prüfen, ob wir in einer gezählten Umgebung sind; 2017-01-06 DK
% auf \ifstr umgestellt, da sonst ggfs. ein Spatium zuviel; 2017-03-04 DK
\newcommand{\head}[2][~]{%
	\pagebreak[1]\vspace{1\baselineskip plus 0.25\baselineskip minus 0.25\baselineskip}%
	\ifnumbering\pstart\else\par\fi%
	{\centering\noindent%
		\ifstr{#1}{~}{\relax}{\page{#1}\ }%
		#2%
	\par}%
	\ifnumbering\pend\fi%
	\nopagebreak[4]\vspace{0.25\baselineskip plus 0.1\baselineskip}\nopagebreak[4]%
}


% Zwischenüberschriften %
\newcommand{\subhead}[1]{%
	%Abstand von 0.25 auf 0.35 erhöht; 2017-01-14 DK % 
	\pagebreak[1]\vspace{0.35\baselineskip plus 0.2\baselineskip minus 0.05\baselineskip}%
	\pstart\noindent\centering #1%
	%Abstand von 0.1 auf 0.2 erhöht; 2017-01-14 DK %
	\pend\nopagebreak[4]\vspace{0.2\baselineskip}\nopagebreak[4]%
}

% Kapitelüberschriften fangen oben an
\RedeclareSectionCommand[%
	beforeskip=0pt,
	afterskip=0.5\baselineskip plus .15\baselineskip minus .15\baselineskip,
	innerskip=0pt
]{chapter}

% Section etwas näher an den Text rücken; 2017-01-23 DK
\RedeclareSectionCommand[%
	afterskip=1.25ex plus .3ex,
	beforeskip=-3.75ex plus -1ex minus -.2ex
]{section}

% Kapitelnummer mit »Nr.« in eine eigene Zeile
\renewcommand*{\chapapp}{Nr.}
\renewcommand*{\chaptermarkformat}{\thechapter. }

% Einstellungen für die (Nicht-)Ausgabe der Nummern
\setcounter{secnumdepth}{\chapternumdepth}
\renewcommand{\sectionformat}{}
\renewcommand{\thesubsection}{\arabic{subsection}.}
\setcounter{tocdepth}{\sectionnumdepth}

% Font der Überschriften
\renewcommand*{\raggedsection}{\centering}
\addtokomafont{chapter}{\normalfont}
% neue Definition seit KOMA-Script 3.20, bisher nur übersehen; 2017-01-27 DK
\DeclareTOCStyleEntry[entryformat=\normalfont\normalsize]{tocline}{chapter}
\addtokomafont{section}{\normalfont}
\addtokomafont{subsection}{\normalfont\textit}
\addtokomafont{minisec}{\raggedright\normalfont\itshape}
\addtokomafont{subsubsection}{\normalfont}
\addtokomafont{paragraph}{\normalfont}

%
% To give reledmac the chance to guess a correct height for footnotes
%\AtBeginDocument{\maxhnotesX{0.4\textheight}} 

\makeatletter
	\newcommand{\getTextSize}{\f@size}
\makeatother

% Abstand zw. Tabellenzeilen
\renewcommand{\arraystretch}{1.5}

% Nach einem Schrägstrich darf getrennt werden, der Schrägstrich selbst ist auch ein Trennzeichen.
\renewcommand{\slash}{/\penalty\exhyphenpenalty\hspace{0pt}}

%\usepackage{newfile}
\newcommand{\habShortTitles}[1]{\relax}%
	%\addtostream{short}{\noexpand\item #1\noexpand\dotfill\thepage
	%}
%}

%%% habCloser
% TODO debug-Modus mit framebox statt makebox?#
% \noindent vor \makebox; 2016-08-26 DK %
\makeatletter
	\newlength{\@habWidth}
	\setlength{\@habWidth}{0pt}
	\newlength{\@habWidth@temp}
	\newcommand{\habCloser}[3]{%
		\@for\next:=#3\do{%
			\settowidth{\@habWidth@temp}{\next}%
			\ifdim\@habWidth<\@habWidth@temp%
				\setlength{\@habWidth}{\@habWidth@temp}%
			\fi%
		}%
		\@tempswafalse
		\@for\next:=#3\do{%
			\if@tempswa\\\else\@tempswatrue\fi%
			\noindent\makebox[\textwidth][#1]{\makebox[\@habWidth][#2]{\next}}%
		}%
	}
\makeatother

% in edtab/edarray keine übergroßen Abstände zwischen den »Spalten«
\setlength{\edtabcolsep}{0pt}

%\usepackage{pbox}
%\usepackage[normalem]{ulem}

% Platz für die Seitenzahlen im Inhaltsverzeichnis; 2016-10-18 DK
\makeatletter
\renewcommand*{\@pnumwidth}{2.2em}
\makeatother

\makeatletter
\newcommand*{\getlength}[1]{\strip@pt#1}
\makeatother

% Verweise im Register mit Pfeil; 2017-01-10 DK
\renewcommand*{\see}[2]{ →~#1}

% Umbruchzeichen für Marginalien; oder →?; 2017-02-03 DK
\newcommand{\marginBreakMark}{\textgreek{▹}}

% Teilbände zentriert ohne Nummer; Lösung basiert auf M.Kohm, Nr. 2000
\makeatletter
\renewcommand*{\l@part}[2]{%
  \addpenalty{-\@highpenalty}% Seitenumbruch erlauben
  \addvspace{2.25em \@plus\p@}% Abstand einfügen
  \begingroup
    \centering
    \interlinepenalty=10000 % Keinen Seitenumbruch beim Zeilenumbruch.
    \renewcommand*{\numberline}[1]{##1\enskip}
    \usekomafont{partentry}{#1\par}%
    % Seitenzahl lassen wir weg.
  \endgroup
  \penalty\numexpr 20009-\parttocdepth\relax
}
\makeatother
\RedeclareSectionCommand[tocstyle={}]{part}%
\renewcommand*{\addparttocentry}[2]{%
  \IfArgIsEmpty{#1}{%
    \addtocentrydefault{part}{#1}{#2}%
  }{%
    \addtocentrydefault{part}{\protect#1}{#2}%
  }%
}

% \usepackage{filecontents}
% \begin{filecontents*}{style1.xdy}
% (markup-indexentry :open "~n      \subIIIitem " :depth 3)
% \end{filecontents*}
\makeatletter
\newcommand\subIIIitem{\@idxitem \hspace*{60\p@}}
\newlength{\sIIIi}
\setlength{\sIIIi}{70\p@}
\makeatother

% Layout des Registers anpassen; 2017-02-19 DK
\usepackage[itemlayout=relhang,hangindent=7mm]{idxlayout}

% weiteres Inhaltsverzeichnis unterstützen; 2017-02-20 DK
\usepackage{shorttoc}
% shorttoc überschreibt die Darstellung des Headers \ldots :(
% Lösung von Werner auf tex.sx 313398
% patch \shorttableofcontents
\usepackage{regexpatch}
\makeatletter
\xpatchcmd*{\anothertableofcontents}{\uppercase}{}{}{}% Remove all \uppercase
\makeatother
% end of patch

\newcommand{\habEnumItem}[1]{%
	\stepcounter{habEnumPos}%
	\ifnum\value{habEnumPos}>0% This should be > 0 if the initial was 0 or more
		\leditem{\thehabEnumPos}{#1}%
	\else
		\leditem{}{#1}%
	\fi
}
\newcounter{habEnumPos}%
\newenvironment{habEnum}[1][0]{%
	\setcounter{habEnumPos}{#1}%
}{%
	\par%
}


%%%%%%%%%%%%%%%%%%%%%%%%%%%%%%%%%%%%%%%%%%%%%%%%%%%%%%%%%%%%%%%%%%%%%%%%%%%%%%%%%%%%%%%%%%%%%%%%%%%%%%%%%%%%%%%%%%%%%%%
%%% Character Protrusion für Junicode laden (geht automatisch nicht ohne Probleme)
% einige zusätzliche Werte definiert (spitze supplied-Klammern, Expansion)
\DeclareCharacterInheritance
   { encoding = {TU} }
   { A = {À,Á,Â,Ä,Å},
     F = {Ḟ},
     f = {ḟ},
     J = {Ĵ},
     K = {Ķ,Ǩ,Ḱ,Ḳ,Ḵ},
     L = {Ł,Ļ,Ľ,Ḷ,Ḹ,Ḻ,Ḽ},
     r = {ŕ,ŗ,ř,ȑ,ȓ,ṙ,ṛ,ṝ,ṟ},
     T = {Ţ,Ť,Ț,Ṫ,Ṭ,Ṯ,Ṱ},
     t = {ţ,ț,ṫ,ṭ,ṯ,ṱ,ẗ},
     V = {Ṽ,Ṿ},
     v = {ṽ,ṿ},
     W = {Ŵ,Ẁ,Ẃ,Ẅ,Ẇ,Ẉ},
     w = {ŵ,ẁ,ẃ,ẅ,ẇ,ẉ,ẘ},
     X = {Ẋ,Ẍ},
     x = {ẋ,ẍ},
     Y = {Ý,Ŷ,Ÿ,Ȳ,Ẏ,Ỳ,Ỵ,Ỷ,Ỹ},
     y = {ý,ÿ,ŷ,ȳ,ẏ,ỳ,ỵ,ỷ,ỹ},
     : = {/colon.alt},
     ; = {/semicolon.alt},
     ! = {/exclam.alt},
     ? = {/question.alt},
     ‘ = {/quoteleft.alt,/quotedblleft.alt},
     ’ = {/quoteright.alt,/quotedblright.alt}
   }

\SetProtrusion
   [ name     = Junicode-default,
     unit     = 1em ]
   { encoding = {EU1,EU2,TU},
     family   = Junicode }
   {
     A = {34,34},
     Æ = {69,  },
     F = {  ,29},
     J = {18,  },
     K = {  ,33},
     L = {  ,33},
     T = {33,33},
     V = {63,31},
     W = {63,31},
     X = {33,33},
     Y = {30,30},
     f = {  ,-23},
     g = {  ,-9},
     k = {  ,25},
     r = {  ,7},
     t = {  ,6},
     v = {21,21},
     w = {21,21},
     x = {23,23},
     y = {  ,21},
     1 = {105,91},
     4 = {24,59},
     7 = {24,24},
     . = { ,178},
    {,}= { ,153},
     : = { ,146},
     ; = { ,87},
     ! = { ,26},
     ? = { ,37},
     @ = {37,37},
     ~ = {66,83},
    \% = {45,45},
     * = {79,79},
     + = {125,125},
     ( = {78,   },
     ) = {   ,69},
     / = {35,70},
     - = {280,280}, % hyphen
     –  = {280,280}, % endash
     — = {280,280}, % emdash
     ‘ = {96,76},
     ’ = {76,96},%〉 = { , 150},
     〉 = { , 200},% supplied
     ˈ = { , 200}% Expansion
   }
\SetProtrusion
   [ name     = Junicode-it,
     unit     = 1em ]
   { encoding = {EU1,EU2,TU},
     family   = Junicode,
     shape    = {it,sl}  }
   {
     A = {43,  },
     Æ = {70,  },
     T = {43,  },
     f = {  ,-35},
    {,}= { ,153},
     : = { ,110},
     ; = { ,110},
     ? = { ,66},
     & = {  ,62},
    \% = {47,47},
     / = {  ,40},
     - = {41,142}, % hyphen
     – = {41,142}, % endash
     — = {41,142}, % emdash
     ‘ = {84,105},
     ’ = {84,105},
     ( = {66,   },
     ) = {   ,66}
   }
%%% ENDE Character Protrusion

%\KOMAoptions{draft=true}
%\showoutput
% In dieser Reihenfolge funktioniert es
\begin{document}

%\label{edoc_ed000227_fg_kuttenberger_religionsfriede--}
\beginnumbering 
\pstart\relax  \page{F7v}\textsuperscript{\aalph{edoc_ed000227_fg_kuttenberger_religionsfriede--d2e40155}}Toto na sněmu obecném, králem, Jeho Milostí,  na Horách Kutnách drženém, jest zjednáno léta Pá-\page{F8r}ně tisícího čtyrstého LXXX°V° a kralování krále Vladislava léta XIIII°· a zapsáno ve dckách pamětných Václava z Chvojence· CXIX°· léta  tisícího D° prvního.\footnotetextA[1]
		{\label{edoc_ed000227_fg_kuttenberger_religionsfriede--d2e40155}Fehlt Talmb., S. 512, Z. 1; Fürstenb., S. 418, Z. 1: We jméno božie ščasně řkúce amen.}  \pend 
\pstart\vspace{0.25\baselineskip}  Pro chválu\footnotetextA
		{Talmb., S. 512, Z. 1 und Fürstenb., S. 418, Z. 1: \textit{danach folgend} pána.} Boha všemohúcího a pro obecné dobré Království českého, aby každý v svém řádu a spravedlnosti státi mohl, \textsuperscript{\aalph{edoc_ed000227_fg_kuttenberger_religionsfriede--d2e40275}}láska, jednota\footnotetextA[1]
		{\label{edoc_ed000227_fg_kuttenberger_religionsfriede--d2e40275}Ebd., S. 418, Z. 2f.: a láska a jednota.} aby\footnotetextA
		{Fehlt Talmb., S. 512, Z. 2.} zachována byla \textsuperscript{\aalph{edoc_ed000227_fg_kuttenberger_religionsfriede--d2e40313}}ode všech\footnotetextA[1]
		{\label{edoc_ed000227_fg_kuttenberger_religionsfriede--d2e40313}Fehlt Fürstenb., S. 418, Z. 3.}, \textsuperscript{\aalph{edoc_ed000227_fg_kuttenberger_religionsfriede--d2e40345}}léta Božího M° CCCC°  osmdesátého pátého\footnotetextA[1]
		{\label{edoc_ed000227_fg_kuttenberger_religionsfriede--d2e40345}Talmb., S. 512, Z. 3: anno etc. MCCCCLXXXV; Fürstenb., S. 418, Z. 3: a to léta páně lxxxv\textsuperscript{o}.}, v neděli postní\footnotetextA
		{Ebd., S. 418, Z. 3: \textit{danach folgend} jenž slove.} letare, majíc\footnotetextA
		{Ebd., S. 418, Z. 4: majíce.}  sněm obecní všeho Království našeho českého,  knížat, pánuov, rytířstva,\footnotetextA
		{Talmb., S. 512, Z. 4: \textit{danach folgend} i; Fürstenb., S. 418, Z. 4: \textit{danach folgend} a.} měst· v městě \textsuperscript{\aalph{edoc_ed000227_fg_kuttenberger_religionsfriede--d2e40479}}Hoře  Kuthn[ě]\footnotetextA[1]
		{\label{edoc_ed000227_fg_kuttenberger_religionsfriede--d2e40479}Ebd., S. 419, Z. 1: na Horách Kutnách.}, My, Vladislav, z Boží milosti král český\footnotetextA
		{Ebd., S. 419, Z. 1f.: \textit{danach folgend} markrabě Morawský.} etc.,  majíc\footnotetextA
		{Talmb., S. 512, Z. 5 und Fürstenb., S. 419, Z. 2: majíce.} toho\footnotetextA
		{Ebd., S. 419, Z. 2: o to.} péči, v témž svém\footnotetextA
		{Ebd., S. 419, Z. 2: našem.} království\footnotetextA
		{Korrigiert aus: kráovství.} a \textsuperscript{\aalph{edoc_ed000227_fg_kuttenberger_religionsfriede--d2e40592}}svých  poddaných znajíc\footnotetextA[1]
		{\label{edoc_ed000227_fg_kuttenberger_religionsfriede--d2e40592}Talmb., S. 512, Z. 6: znajíce swých poddaných.}\footnotetextA
		{Fürstenb., S. 419, Z. 2: znajíce.} potřebné věci, \textsuperscript{\aalph{edoc_ed000227_fg_kuttenberger_religionsfriede--d2e40621}}tyto artikule\footnotetextA[1]
		{\label{edoc_ed000227_fg_kuttenberger_religionsfriede--d2e40621}Ebd., S. 419, Z. 3: artikule dolepsané.}  zřídili a zpuosobili a \textsuperscript{\aalph{edoc_ed000227_fg_kuttenberger_religionsfriede--d2e40641}}na konec\footnotetextA[1]
		{\label{edoc_ed000227_fg_kuttenberger_religionsfriede--d2e40641}Ebd., S. 419, Z. 3: konečně.} zavřeli jsme.  \pend 
\pstart\vspace{0.25\baselineskip}  Najprvé,\footnotetextA
		{Talmb., S. 512, Z. 8 und Fürstenb., S. 419, Z. 4: Item najprwé.} což se víry\footnotetextA
		{Ebd., S. 419, Z. 4: \textit{danach folgend} dotýče.} přijímání\footnotetextA
		{Korrigiert aus: přijíání.} Těla a Krve Pána \textsuperscript{\aalph{edoc_ed000227_fg_kuttenberger_religionsfriede--d2e40739}}Krista Ježíše\footnotetextA[1]
		{\label{edoc_ed000227_fg_kuttenberger_religionsfriede--d2e40739}Korrigiert nach Talmb., S. 512, Z. 8: Ježíše Krista; Fürstenb., S. 419, Z. 4: Ježíše Christa. \textit{Textvorlage:} Krista Jejíše.} pod jednou neb pod obojí zpuosobú dotýče\footnotetextA
		{Ebd., S. 419, Z. 5: rozdáwanie.}, král \textsuperscript{\aalph{edoc_ed000227_fg_kuttenberger_religionsfriede--d2e40806}}Jeho Milost\footnotetextA[1]
		{\label{edoc_ed000227_fg_kuttenberger_religionsfriede--d2e40806}Talmb., S. 512, Z. 9 und Fürstenb., S. 419, Z. 5: \textit{aufgrund fehlerhafter Transkription} JM\textsuperscript{t} \textit{statt} geho miloſt.} \textsuperscript{\aalph{edoc_ed000227_fg_kuttenberger_religionsfriede--d2e40831}}to rozváživ\footnotetextA[1]
		{\label{edoc_ed000227_fg_kuttenberger_religionsfriede--d2e40831}Ebd., S. 419, Z. 5: rozwážie.}, za slušné\textsuperscript{\aalph{edoc_ed000227_fg_kuttenberger_religionsfriede--d2e40867}}Jeho Milosti zdá se v tom toto\footnotetextA[1]
		{\label{edoc_ed000227_fg_kuttenberger_religionsfriede--d2e40867}Talmb., S. 512, Z. 9f.: \textit{aufgrund fehlerhafter Transkription} se w tom JM\textsuperscript{ti} zdá toto \textit{statt} se w tom geho milosti zdá toto; Fürstenb., S. 419, Z. 5: zdá se JM\textsuperscript{ti} w tom toto.}, aby strana strany  nehaněla ani utiskala, \textsuperscript{\aalph{edoc_ed000227_fg_kuttenberger_religionsfriede--d2e40916}}buď z\footnotetextA[1]
		{\label{edoc_ed000227_fg_kuttenberger_religionsfriede--d2e40916}Talmb., S. 512, Z. 10: buď; Fürstenb., S. 419, Z. 6: buďto.} světských, neb duchovních: než obojí\footnotetextA
		{Ebd., S. 419, Z. 6: \textit{danach folgend} aby.} k sobě lásku zachovajte\footnotetextA
		{Ebd., S. 419, Z. 7: zachowali.}, kněží  kterékoli strany, pod kterýmikoli knížaty, pány, rytířstvem\footnotetextA
		{Ebd., S. 419, Z. 7: \textit{danach folgend} a.}, městy\footnotetextA
		{Talmb., S. 512, Z. 12: \textit{danach folgend} králowskými.} sú\footnotetextA
		{Fürstenb., S. 419, Z. 7: \textit{danach folgend} ti.}, slovo Boží svobodně \textsuperscript{\aalph{edoc_ed000227_fg_kuttenberger_religionsfriede--d2e41084}}a na hříchy kažte\footnotetextA[1]
		{\label{edoc_ed000227_fg_kuttenberger_religionsfriede--d2e41084}Ebd., S. 419, Z. 8: kažte na hřiechy.},  \page{F8v}\footnotetextA
		{Ebd., S. 419, Z. 8: \textit{zusätzlich} a aby.} žádný žádných \textsuperscript{\aalph{edoc_ed000227_fg_kuttenberger_religionsfriede--d2e41119}}nekaceřuje ani haněje\footnotetextA[1]
		{\label{edoc_ed000227_fg_kuttenberger_religionsfriede--d2e41119}Talmb., S. 512, Z. 13: nekaceřujte ani hanějte; Fürstenb., S. 419, Z. 8: nekaceřoval ani haněl.}. A knížata, páni, rytířstvo i v městech \textsuperscript{\aalph{edoc_ed000227_fg_kuttenberger_religionsfriede--d2e41172}}královskych ti\footnotetextA[1]
		{\label{edoc_ed000227_fg_kuttenberger_religionsfriede--d2e41172}Ebd., S. 419, Z. 9: králowstwie našeho.}, kteří  obyčej mají přijímati Tělo a Krev Boží\footnotetextA
		{Talmb., S. 512, Z. 14: pána Krista.} pod jednú zpuosobú\footnotetextA
		{Fürstenb., S. 419, Z. 10: \textit{danach folgend} majíce také.}, kněží \textsuperscript{\aalph{edoc_ed000227_fg_kuttenberger_religionsfriede--d2e41247}}i lidi a\footnotetextA[1]
		{\label{edoc_ed000227_fg_kuttenberger_religionsfriede--d2e41247}Ebd., S. 419, Z. 10: a lidi i.} poddané své\footnotetextA
		{Talmb., S. 512, Z. 15: \textit{danach folgend} kteréž.} pod sebú\footnotetextA
		{Fürstenb., S. 419, Z. 10: \textit{danach folgend} ježto.} májí obyčej, přijímajíce\footnotetextA
		{Talmb., S. 512, Z. 16: přijímající; Fürstenb., S. 419, Z. 11: přijímati.} Tělo a Krev Pána Krista  pod obojí zpuosobú\footnotetextA
		{Ebd., S. 419, Z. 11: \textit{danach folgend} w tom.}, žádných útiskuov nečiňte\footnotetextA
		{Ebd., S. 419, Z. 11: sobě nečiniece.}, ani  jich\footnotetextA
		{Ebd., S. 419, Z. 12: sebe.} haňte\footnotetextA
		{Korrigiert aus: hoňte.}\footnotetextA
		{Talmb., S. 512, Z. 17: hanějte; Fürstenb., S. 419, Z. 12: hanějíce.}, ani jim braňte\footnotetextA
		{Ebd., S. 419, Z. 12: bráníce.} svého\footnotetextA
		{Ebd., S. 419, Z. 12: jich.} spasení hledati podlé jich \textsuperscript{\aalph{edoc_ed000227_fg_kuttenberger_religionsfriede--d2e41446}}víry a zvyklosti\footnotetextA[1]
		{\label{edoc_ed000227_fg_kuttenberger_religionsfriede--d2e41446}Talmb., S. 512, Z. 17: zwyklosti a wíry.}. Též i ti páni\footnotetextA
		{Ebd., S. 512, Z. 18: \textit{danach folgend} a.}, rytířstvo  i\footnotetextA
		{Fehlt ebd., S. 512, Z. 18; Fürstenb., S. 419, Z. 13: a.} města, kteříž obyčej mají přijímati Tělo a·  Krev Pána Krista pod obojí zpuosobú· a mají poddané své a kněží: ješto \textsuperscript{\aalph{edoc_ed000227_fg_kuttenberger_religionsfriede--d2e41569}}obyčej mají\footnotetextA[1]
		{\label{edoc_ed000227_fg_kuttenberger_religionsfriede--d2e41569}Talmb., S. 512, Z. 18 und Fürstenb., S. 419, Z. 13: mají obyčej.} přijímati pod jednou zpuosobú: k těm poddaným\footnotetextA
		{Korrigiert aus: peddaným.} svým a  proti ním \textsuperscript{\aalph{edoc_ed000227_fg_kuttenberger_religionsfriede--d2e41636}}mají se též\footnotetextA[1]
		{\label{edoc_ed000227_fg_kuttenberger_religionsfriede--d2e41636}Talmb., S. 512, Z. 20: se mají.} zachovati bez útisku\footnotetextA
		{Ebd., S. 512, Z. 21: útiskuow.}, aby každý \textsuperscript{\aalph{edoc_ed000227_fg_kuttenberger_religionsfriede--d2e41678}}slovo Boží\footnotetextA[1]
		{\label{edoc_ed000227_fg_kuttenberger_religionsfriede--d2e41678}Ebd., S. 512, Z. 21: boží slowo.} kázal bez hanění na hříchy a svého  spasení hledati mohl: podlé víry své a zvyklosti  bez vyhánění\footnotetextA
		{Fürstenb., S. 419, Z. 17: haněnie.} a útisku\footnotetextA
		{Talmb., S. 512, Z. 22 und Fürstenb., S. 419, Z. 17: \textit{aufgrund fehlerhafter Transkription} útiskuow \textit{statt} autiſkuow.}. A to státi má a zachováno býti bez pohnutí oběma stranoma až do \textsuperscript{\aalph{edoc_ed000227_fg_kuttenberger_religionsfriede--d2e41802}}let třidceti a jednoho\footnotetextA[1]
		{\label{edoc_ed000227_fg_kuttenberger_religionsfriede--d2e41802}Talmb., S. 512, Z. 23: let XXXI; Fürstenb., S. 419, Z. 18: do xxxi léta.}: od nynějšího času pořád zběhlého\footnotetextA
		{Talmb., S. 512, Z. 24: zběhlých.}. A což se \textsuperscript{\aalph{edoc_ed000227_fg_kuttenberger_religionsfriede--d2e41872}}dotýče kompaktát a smlúvy koncilium  bazilejského\footnotetextA[1]
		{\label{edoc_ed000227_fg_kuttenberger_religionsfriede--d2e41872}Fürstenb., S. 419, Z. 18f.: kompaktat Basilejského dotýče.}·, ty na své míře stuojte a v své moci zuostaňte, jak samy\footnotetextA
		{Fehlt ebd., S. 419, Z. 19.} v sobě jsú: a\footnotetextA
		{Fehlt ebd., S. 419, Z. 20.} v tom času  páni, rytířstvo i města obyčej\footnotetextA
		{Talmb., S. 512, Z. 26: obyčeje.} majíce přijímati Tělo a Krev Pána Krista pod obojí zpuosobú: mají, bude-li se jim zdáti, své poselství učiniti k Otci svatému, a knížata, páni\footnotetextA
		{Ebd., S. 512, Z. 28: \textit{danach folgend} a.}, rytířstvo i města \textsuperscript{\aalph{edoc_ed000227_fg_kuttenberger_religionsfriede--d2e42142}}strany· obyčej majíc\footnotetextA[1]
		{\label{edoc_ed000227_fg_kuttenberger_religionsfriede--d2e42142}Ebd., S. 512, Z. 28: obyčeje majíce; Fürstenb., S. 419, Z. 22: z strany obyčej majíc.} přijímati Tělo a Krev Pána /  \page{G1r}Krista pod jednú zpuosobú, též podlé nich mají svú  pilnost i\footnotetextA
		{Talmb., S. 512, Z. 29: a.} péči učiniti· též\footnotetextA
		{Fehlt ebd., S. 512, Z. 29 und Fürstenb., S. 419, Z. 23.} k Otci svatému, aby slušné cesty \textsuperscript{\aalph{edoc_ed000227_fg_kuttenberger_religionsfriede--d2e42259}}mohly nalezeny\footnotetextA[1]
		{\label{edoc_ed000227_fg_kuttenberger_religionsfriede--d2e42259}Talmb., S. 513, Z. 1: nalezeny mohly.} býti u \textsuperscript{\aalph{edoc_ed000227_fg_kuttenberger_religionsfriede--d2e42278}}Jeho Svatosti\footnotetextA[1]
		{\label{edoc_ed000227_fg_kuttenberger_religionsfriede--d2e42278}Ebd., S. 513, Z. 1: \textit{aufgrund fehlerhafter Transkription} JS\textsuperscript{ti} \textit{statt} geho Swatoſti.}, kteréž by na věčnost \textsuperscript{\aalph{edoc_ed000227_fg_kuttenberger_religionsfriede--d2e42320}}mezi stranami\footnotetextA[1]
		{\label{edoc_ed000227_fg_kuttenberger_religionsfriede--d2e42320}Fehlt Fürstenb., S. 419, Z. 24.} zachovány býti  mohly. Král \textsuperscript{\aalph{edoc_ed000227_fg_kuttenberger_religionsfriede--d2e42352}}Jeho Milost\footnotetextA[1]
		{\label{edoc_ed000227_fg_kuttenberger_religionsfriede--d2e42352}Talmb., S. 513, Z. 2 und Fürstenb., S. 419, Z. 25: \textit{aufgrund fehlerhafter Transkription} JM\textsuperscript{t} \textit{statt} geho milost.} svú\footnotetextA
		{Fehlt ebd., S. 419, Z. 25.} pilnost\footnotetextA
		{Talmb., S. 513, Z. 2 und Fürstenb., S. 419, Z. 25: \textit{danach folgend} i.}, péči, též\footnotetextA
		{Fehlt Talmb., S. 513, Z. 2 und Fürstenb., S. 419, Z. 25.} při Otci svatém \textsuperscript{\aalph{edoc_ed000227_fg_kuttenberger_religionsfriede--d2e42451}}skrze sám\footnotetextA[1]
		{\label{edoc_ed000227_fg_kuttenberger_religionsfriede--d2e42451}Talmb., S. 513, Z. 2f.: sám skrze.} se \textsuperscript{\aalph{edoc_ed000227_fg_kuttenberger_religionsfriede--d2e42468}}Jeho Milost\footnotetextA[1]
		{\label{edoc_ed000227_fg_kuttenberger_religionsfriede--d2e42468}Ebd., S. 513, Z. 3 und Fürstenb., S. 419, Z. 25: \textit{aufgrund fehlerhafter Transkription} JM\textsuperscript{t} \textit{statt} geho milost.} i přátely\footnotetextA
		{Talmb., S. 513, Z. 3: přátele.} své \textsuperscript{\aalph{edoc_ed000227_fg_kuttenberger_religionsfriede--d2e42532}}světské i duchovní míti\footnotetextA[1]
		{\label{edoc_ed000227_fg_kuttenberger_religionsfriede--d2e42532}Ebd., S. 513, Z. 3: duchowní i swětské činiti; Fürstenb., S. 419, Z. 25f.: duchownie i swětské mieti.} ráčí, aby \textsuperscript{\aalph{edoc_ed000227_fg_kuttenberger_religionsfriede--d2e42561}}toho dosaženo býti  mohlo\footnotetextA[1]
		{\label{edoc_ed000227_fg_kuttenberger_religionsfriede--d2e42561}Talmb., S. 513, Z. 3f.: to mohlo dosaženo býti; Fürstenb., S. 419, Z. 26: toho dosaženo mohlo býti.}: aby \textsuperscript{\aalph{edoc_ed000227_fg_kuttenberger_religionsfriede--d2e42579}}Jeho Milosti\footnotetextA[1]
		{\label{edoc_ed000227_fg_kuttenberger_religionsfriede--d2e42579}Talmb., S. 513, Z. 4 und Fürstenb., S. 419, Z. 26: \textit{aufgrund fehlerhafter Transkription} JM\textsuperscript{ti} \textit{statt} geho miloſti.} poddaní na věčnost \textsuperscript{\aalph{edoc_ed000227_fg_kuttenberger_religionsfriede--d2e42637}}v lásce a bez útisku mezi sebú o to býti\footnotetextA[1]
		{\label{edoc_ed000227_fg_kuttenberger_religionsfriede--d2e42637}Ebd., S. 419, Z. 26f.: a bez útiskuow o to trvati.} mohli. A to poselství k Otci svatému má posláno\footnotetextA
		{Talmb., S. 513, Z. 5: \textit{danach folgend} býti.} a vypraveno býti\footnotetextA
		{Fehlt ebd., S. 513, Z. 5.}, což bude moci býti\footnotetextA
		{Fehlt ebd., S. 513, Z. 5 und Fürstenb., S. 419, Z. 27f.} najspíše\footnotetextA
		{Talmb., S. 513, Z. 5f.: najspíš.}. A král \textsuperscript{\aalph{edoc_ed000227_fg_kuttenberger_religionsfriede--d2e42762}}Jeho Milost\footnotetextA[1]
		{\label{edoc_ed000227_fg_kuttenberger_religionsfriede--d2e42762}Ebd., S. 513, Z. 6 und Fürstenb., S. 419, Z. 28: \textit{aufgrund fehlerhafter Transkription} JM\textsuperscript{t} \textit{statt} geho miłoſt.} \textsuperscript{\aalph{edoc_ed000227_fg_kuttenberger_religionsfriede--d2e42789}}svú pilnost\footnotetextA[1]
		{\label{edoc_ed000227_fg_kuttenberger_religionsfriede--d2e42789}Fehlt ebd., S. 419, Z. 28.} jako král a pán náš\footnotetextA
		{Ebd., S. 419, Z. 28: \textit{danach folgend} milostiwý.} ráčí v tom míti\footnotetextA
		{Talmb., S. 513, Z. 6: činiti; Fürstenb., S. 419, Z. 28: swú pilnost mieti.}, aby se to nedlilo a neprodlévalo.  \pend 
\pstart\vspace{0.25\baselineskip}  Item jakož před časy minulými smlúva se jest a zavření stalo mezi stranami obojími: a cedule toho\footnotetextA
		{Fehlt Talmb., S. 513, Z. 8f.} vyřezané \textsuperscript{\aalph{edoc_ed000227_fg_kuttenberger_religionsfriede--spcrit7}}smluvce\footnotetextA
		{Ebd., S. 513, Z. 9: smluwčí.} před rukama\textsuperscript{\aalph{edoc_ed000227_fg_kuttenberger_religionsfriede--spcrit7}}\begin{marginFoot}\footnotetextA[1]
		{\edlabel{edoc_ed000227_fg_kuttenberger_religionsfriede--spcrit7}\label{edoc_ed000227_fg_kuttenberger_religionsfriede--spcrit7}Fehlt , Fürstenb., S. 419, Z. 31.}\end{marginFoot}· jsú \textsuperscript{\aalph{edoc_ed000227_fg_kuttenberger_religionsfriede--d2e42977}}pro znamení\footnotetextA[1]
		{\label{edoc_ed000227_fg_kuttenberger_religionsfriede--d2e42977}Ebd., S. 513, Z. 9 und Fürstenb., S. 419, Z. 31: i známe.},  že se jest \textsuperscript{\aalph{edoc_ed000227_fg_kuttenberger_religionsfriede--d2e43010}}ještě tomu\footnotetextA[1]
		{\label{edoc_ed000227_fg_kuttenberger_religionsfriede--d2e43010}Talmb., S. 513, Z. 10: tomu ještě; Fürstenb., S. 419, Z. 32: tomu.} dosti nestalo. Protož to se nám  zdá, aby se tomu dosti stalo a napraveno bylo\footnotetextA
		{Ebd., S. 419, Z. 32: \textit{danach folgend} tak.}: jakož  ty cedule \textsuperscript{\aalph{edoc_ed000227_fg_kuttenberger_religionsfriede--d2e43090}}ukazují a samy v sobě\footnotetextA[1]
		{\label{edoc_ed000227_fg_kuttenberger_religionsfriede--d2e43090}Ebd., S. 419, Z. 32: w sobě ukazují a.} zavírají s obú stranú \textsuperscript{\aalph{edoc_ed000227_fg_kuttenberger_religionsfriede--d2e43146}}do suchých dní, najprvé příštích, po hodu  Ducha svatého\footnotetextA[1]
		{\label{edoc_ed000227_fg_kuttenberger_religionsfriede--d2e43146}Talmb., S. 513, Z. 11f.: do hodu swatého Wácslawa najprwé přištieho; Fürstenb., S. 419, Z. 33: a to konečně o suchých dnech najprv příštích po hodu slawném Ducha swatého.}.  \pend 
\pstart\vspace{0.25\baselineskip}  Item což se Prahy dotýče, při té smlúvě, kterúž  král \textsuperscript{\aalph{edoc_ed000227_fg_kuttenberger_religionsfriede--d2e43211}}Jeho Milost\footnotetextA[1]
		{\label{edoc_ed000227_fg_kuttenberger_religionsfriede--d2e43211}Talmb., S. 513, Z. 13 und Fürstenb., S. 419, Z. 34: \textit{aufgrund fehlerhafter Transkription} JM\textsuperscript{t} \textit{statt} geho miłost.} s nimi má a oni \textsuperscript{\aalph{edoc_ed000227_fg_kuttenberger_religionsfriede--d2e43252}}s Jeho Milostí,  Jeho Milost\footnotetextA[1]
		{\label{edoc_ed000227_fg_kuttenberger_religionsfriede--d2e43252}Talmb., S. 513, Z. 14 und Fürstenb., S. 419, Z. 34f.: \textit{aufgrund fehlerhafter Transkription} s JM\textsuperscript{tí}, JM\textsuperscript{t} \textit{statt} geho miłosti, geho miłost.} jich nechati\footnotetextA
		{Ebd., S. 419, Z. 35: \textit{danach folgend} přitom.} ráčí, aby ti páni rytíř-\page{G1v}stvo i\footnotetextA
		{Ebd., S. 419, Z. 35: a.} města, kteřížto mezi \textsuperscript{\aalph{edoc_ed000227_fg_kuttenberger_religionsfriede--d2e43366}}Jeho královskú Milostí\footnotetextA[1]
		{\label{edoc_ed000227_fg_kuttenberger_religionsfriede--d2e43366}Talmb., S. 513, Z. 14f.: \textit{aufgrund fehlerhafter Transkription} JM\textsuperscript{tí} králowſkú \textit{statt} geho miłoſti krałowſku; Fürstenb., S. 419, Z. 35: JKM\textsuperscript{ti}.} a Pražany jednati mají· o\footnotetextA
		{Ebd., S. 419, Z. 36: a.} to, což jest v těch /  smlúvách pozústaveno, to jednali: aby \textsuperscript{\aalph{edoc_ed000227_fg_kuttenberger_religionsfriede--d2e43457}}to, což se jest\footnotetextA[1]
		{\label{edoc_ed000227_fg_kuttenberger_religionsfriede--d2e43457}Ebd., S. 419, Z. 36: se tomu dosti.}  stalo, k slušné nápravě přišlo\footnotetextA
		{Fehlt ebd., S. 419, Z. 37.}, a to hned první  pondělí po provodní neděli\footnotetextA
		{Talmb., S. 513, Z. 17f.: \textit{danach folgend} ti páni, rytieřstwo i města aby se do Prahy sjeli, a to před se wzali a jednali, aby o to konec byl.; Fürstenb., S. 419, Z. 37f.: \textit{danach folgend} ti páni a rytieřstwo i města do Prahy se sjeti mají a to předse wzieti a jednati, aby o to konec byl.}. \textsuperscript{\aalph{edoc_ed000227_fg_kuttenberger_religionsfriede--d2e43563}}Pakli by se  přihodilo, že by o to konec učiněn byl,\footnotetextA[1]
		{\label{edoc_ed000227_fg_kuttenberger_religionsfriede--d2e43563}Ebd., S. 419, Z. 38f.: Pakliby o to konce se nestalo.}\footnotetextA
		{Talmb., S. 513, Z.19: nebyl.} tehda\footnotetextA
		{Ebd., S. 513, Z. 19 und Fürstenb., S. 419, Z. 39: tehdy.} knížata, páni, rytířstvo i města \textsuperscript{\aalph{edoc_ed000227_fg_kuttenberger_religionsfriede--d2e43626}}z strany\footnotetextA[1]
		{\label{edoc_ed000227_fg_kuttenberger_religionsfriede--d2e43626}Talmb., S. 513, Z. 19: \textit{aufgrund fehlerhafter Transkription} stranný \textit{statt} ſtrany.}: obyčej majíce přijímati pod jednou zpuosobou\footnotetextA
		{Fürstenb., S. 419, Z. 40: \textit{danach folgend} přijímati.}, budú moci  o to mluviti i státi podlé vší potřebnosti, ač\footnotetextA
		{Talmb., S. 513, Z. 20: až.} se jim  bude zdáti, aby \textsuperscript{\aalph{edoc_ed000227_fg_kuttenberger_religionsfriede--d2e43744}}to tak k slušné nápravě\footnotetextA[1]
		{\label{edoc_ed000227_fg_kuttenberger_religionsfriede--d2e43744}Ebd., S. 513, Z. 21: k slušné nápravě; Fürstenb., S. 419, Z. 41: o to slušně naprawenie.} \textsuperscript{\aalph{edoc_ed000227_fg_kuttenberger_religionsfriede--spcrit1}}vždy přišlo  a napraveno bylo konečně \textsuperscript{\aalph{edoc_ed000227_fg_kuttenberger_religionsfriede--spcrit3}}do hodu svatého Vaclava\textsuperscript{\aalph{edoc_ed000227_fg_kuttenberger_religionsfriede--spcrit3}}\begin{marginFoot}\footnotetextA[1]
		{\edlabel{edoc_ed000227_fg_kuttenberger_religionsfriede--spcrit3}\label{edoc_ed000227_fg_kuttenberger_religionsfriede--spcrit3}, Fürstenb., S. 419, Z. 42: do sw. Wácslawa.}\end{marginFoot} najprv příštího\textsuperscript{\aalph{edoc_ed000227_fg_kuttenberger_religionsfriede--spcrit1}}\begin{marginFoot}\footnotetextA[1]
		{\edlabel{edoc_ed000227_fg_kuttenberger_religionsfriede--spcrit1}\label{edoc_ed000227_fg_kuttenberger_religionsfriede--spcrit1}, Talmb., S. 513, Z. 21: přišlo.}\end{marginFoot}.  Item poněvádž mnohá jednání a sjezdové bývali jsú \textsuperscript{\aalph{edoc_ed000227_fg_kuttenberger_religionsfriede--d2e43858}}o to;\footnotetextA[1]
		{\label{edoc_ed000227_fg_kuttenberger_religionsfriede--d2e43858}Fehlt Talmb., S. 513, Z. 22 und Fürstenb., S. 419, Z. 43.}a potom vždy strkové\footnotetextA
		{Talmb., S. 513, Z. 23: ustrkowé; Fürstenb., S. 419, Z. 43: postrkowé.} a nezachování,  zdá se nám a chceme\footnotetextA
		{Ebd., S. 419, Z. 44: \textit{danach folgend} tomu.}: aby všickni\footnotetextA
		{Talmb., S. 513, Z. 23: wšichni.} obyvatelé Království tohoto českého\footnotetextA
		{Fehlt ebd., S. 513, Z. 23 und Fürstenb., S. 419, Z. 44.}, knížata, páni, rytířstvo\footnotetextA
		{Talmb., S. 513, Z. 24 und Fürstenb., S. 420, Z. 1: \textit{danach folgend} i.},  města jeden zavázek a zápis na to.\footnotetextA
		{Ebd., S. 420, Z. 1: učinili.} \textsuperscript{\aalph{edoc_ed000227_fg_kuttenberger_religionsfriede--d2e44054}}A těch zápisúv  dva listy, aby každá strana jeden v své moci měla,\footnotetextA[1]
		{\label{edoc_ed000227_fg_kuttenberger_religionsfriede--d2e44054}Mit Klammern in Talmb., S. 513, Z. 24f. und Fürstenb., S. 420, Z. 1f.}  podlé tohoto\footnotetextA
		{Ebd., S. 420, Z. 2: toho.} přípisu dole psaného učinili\footnotetextA
		{Fehlt ebd., S. 420, Z. 2.}: a se zapsali\footnotetextA
		{Ebd., S. 420, Z. 2f.: zapsati mají.} do těch \textsuperscript{\aalph{edoc_ed000227_fg_kuttenberger_religionsfriede--spcrit38}}let třidceti\footnotetextA
		{Talmb., S. 513, Z. 26: třidcíti.} a jednoho\textsuperscript{\aalph{edoc_ed000227_fg_kuttenberger_religionsfriede--spcrit38}}\begin{marginFoot}\footnotetextA[1]
		{\edlabel{edoc_ed000227_fg_kuttenberger_religionsfriede--spcrit38}\label{edoc_ed000227_fg_kuttenberger_religionsfriede--spcrit38}Fürstenb., S. 420, Z. 3: xxxi let.}\end{marginFoot}: a pod kterými\footnotetextA
		{Fürstenb., S. 420, Z. 3: těmi.} pokutami, jakž přípis toho okazuje, a\footnotetextA
		{Ebd., S. 420, Z. 3: \textit{danach folgend} to.} oboje strána společní\footnotetextA
		{Talmb., S. 513, Z. 27: společně; fehlt Fürstenb., S. 420, Z. 4.} má a \textsuperscript{\aalph{edoc_ed000227_fg_kuttenberger_religionsfriede--d2e44234}}povinna bude\footnotetextA[1]
		{\label{edoc_ed000227_fg_kuttenberger_religionsfriede--d2e44234}Ebd., S. 420, Z. 4: powinni budú.} udělati a ihned\footnotetextA
		{Talmb., S. 513, Z. 28: hned.}  obojí\footnotetextA
		{Ebd., S. 513, Z. 28: \textit{danach folgend} strana.} před námi nyní\footnotetextA
		{Fehlt ebd., S. 513, Z. 28.} mají sobě slíbiti to zdržeti a  zachovati až do sjezdu a zápisu, \textsuperscript{\aalph{edoc_ed000227_fg_kuttenberger_religionsfriede--d2e44329}}kterýž, dá-li Buoh\footnotetextA[1]
		{\label{edoc_ed000227_fg_kuttenberger_religionsfriede--d2e44329}Ebd., S. 513, Z. 29: kterýž bohdá; Fürstenb., S. 420, Z. 5: když bóh dá.}, při  času hodu \textsuperscript{\aalph{edoc_ed000227_fg_kuttenberger_religionsfriede--d2e44358}}svatého Ducha\footnotetextA[1]
		{\label{edoc_ed000227_fg_kuttenberger_religionsfriede--d2e44358}Talmb., S. 513, Z. 29: \textit{aufgrund fehlerhafter Transkription} Ducha sw. \textit{statt} Ducha swatého; Fürstenb., S. 420, Z. 5: sw. Ducha.} o suchých dnech nyní příštích sněm držeti chceme\footnotetextA
		{Talmb., S. 513, Z. 30: chcem.}: a ten zápis dokonati\footnotetextA
		{Fürstenb., S. 420, Z. 6: \textit{danach folgend} w tato slowa.}.  \pend 
\pstart\vspace{0.25\baselineskip}  \page{G2r}\textgreek{ℂ} \textsuperscript{\aalph{edoc_ed000227_fg_kuttenberger_religionsfriede--d2e44451}}Zápis vo artikule předepsané na cedulech.\footnotetextA[1]
		{\label{edoc_ed000227_fg_kuttenberger_religionsfriede--d2e44451}Fehlt in Talmb., S. 513, Z. 31 und Fürstenb., S. 420, Z. 7.}\pend 
\pstart\vspace{0.25\baselineskip}  Ve jméno Boží Amen. My, Vladislav, z Boží milosti král český \textsuperscript{\aalph{edoc_ed000227_fg_kuttenberger_religionsfriede--d2e44521}}a markrabě\footnotetextA[1]
		{\label{edoc_ed000227_fg_kuttenberger_religionsfriede--d2e44521}Talmb., S. 513, Z. 31 und Fürstenb., S. 420, Z. 7: markrabie.} moravský etc. všem  nynějším i budúcím k věčné památce\footnotetextA
		{Talmb., S. 513, Z. 32 und Fürstenb., S. 420, Z. 8: paměti.} přivodíme: že jsme našli v království našem ruoznici  mezi knížaty, pány, rytířstvem a\footnotetextA
		{Talmb., S. 513, Z. 33: i.} městy, poddanými a věrnými našimi, kterážto i za předkuov našich ještě byla povznikla i často vznikala. Ač\footnotetextA
		{Ebd., S. 513, Z. 35: \textit{aufgrund fehlerhafter Transkription} až \textit{statt} ač.}  i předky našimi, i též předky jich, pány a rytířstvem i městy častokrát srovnávána\footnotetextA
		{Ebd., S. 513, Z. 36: srovnána.} bývala a  na slušných měrách\footnotetextA
		{Fürstenb., S. 420, Z. 12: miestech.} postavena, ale potom opět  v ruoznici\footnotetextA
		{Korrigiert aus: v ruozniei.} vcházela, jakožto\footnotetextA
		{Talmb., S. 513, Z. 37: jakož.} toho\footnotetextA
		{Fehlt Fürstenb., S. 420, Z. 12.} listové i bulle \textsuperscript{\aalph{edoc_ed000227_fg_kuttenberger_religionsfriede--d2e44886}}jsú  Otce s[vatéh]o\footnotetextA[1]
		{\label{edoc_ed000227_fg_kuttenberger_religionsfriede--d2e44886}Ebd., S. 420, Z. 12: otce sw.} papeže Eugenia a zboru neb koncilium bazilejského i slavné paměti císařuov římských a\footnotetextA
		{Ebd., S. 420, Z. 14: i.}  králuov\footnotetextA
		{Korrigiert aus: ktáluov.} českých, předkuov našich milých:\footnotetextA
		{Talmb., S. 514, Z. 2: \textit{danach folgend} i.} tudíž  mnohé zápisy\footnotetextA
		{Ebd., S. 514, Z. 2: \textit{danach folgend} rozličné a.} \textsuperscript{\aalph{edoc_ed000227_fg_kuttenberger_religionsfriede--d2e44997}}obecné a společné\footnotetextA[1]
		{\label{edoc_ed000227_fg_kuttenberger_religionsfriede--d2e44997}Fürstenb., S. 420, Z. 14: společnie a obecné.} předkuov pánuov,  rytířstva i měst nyní dotčených. I zdálo se nám  najužitečnější \textsuperscript{\aalph{edoc_ed000227_fg_kuttenberger_religionsfriede--d2e45059}}i péče té\footnotetextA[1]
		{\label{edoc_ed000227_fg_kuttenberger_religionsfriede--d2e45059}Talmb., S. 514, Z. 3f.: z té péče; Fürstenb., S. 420, Z. 15: z péče té.} a milosti, kterúž máme ke  všem obyvatelom již jmenovaným, pánom, rytířstvu i městom: abychom jeden obecný sjezd vší \textsuperscript{\aalph{edoc_ed000227_fg_kuttenberger_religionsfriede--d2e45159}}země naší\footnotetextA[1]
		{\label{edoc_ed000227_fg_kuttenberger_religionsfriede--d2e45159}Talmb., S. 514, Z. 5: zemi našeho; Fürstenb., S. 420, Z. 17: země.} království tohoto položili \textsuperscript{\aalph{edoc_ed000227_fg_kuttenberger_religionsfriede--d2e45186}}a při tom\footnotetextA[1]
		{\label{edoc_ed000227_fg_kuttenberger_religionsfriede--d2e45186}Ebd., S. 420, Z. 17: při kterémžto.} sněmu z daru Pána Boha všemohúcího s jich obapolní\footnotetextA
		{Ebd., S. 420, Z. 18: obapolnú.}  vúlí\footnotetextA
		{Talmb., S. 514, Z. 6 und Fürstenb., S. 420, Z. 18: \textit{danach folgend} a wědomím.} o ty rúznice všeckny\footnotetextA
		{Talmb., S. 514, Z. 7 und Fürstenb., S. 420, Z. 18: wšecky.}, kteréž byly vznikly, sme srovnali a k místu přivedli, Jakož toho srovnání\footnotetextA
		{Talmb., S. 514, Z. 8: zjednání; Fürstenb., S. 420, Z. 19: jednánie.} \textsuperscript{\aalph{edoc_ed000227_fg_kuttenberger_religionsfriede--d2e45351}}a výpovědi\footnotetextA[1]
		{\label{edoc_ed000227_fg_kuttenberger_religionsfriede--d2e45351}Ebd., S. 420, Z. 19: wýpowěď.} každá strana \textsuperscript{\aalph{edoc_ed000227_fg_kuttenberger_religionsfriede--d2e45368}}pod naší pečetí\footnotetextA[1]
		{\label{edoc_ed000227_fg_kuttenberger_religionsfriede--d2e45368}Fehlt ebd., S. 420, Z. 19f.} pamět má,  v kterémžto srovnání najznamenitější jeden artikul napřed položen jest:\pend 
\pstart\vspace{0.25\baselineskip}  \page{G2v}Což se přijímání Těla a Krve \textsuperscript{\aalph{edoc_ed000227_fg_kuttenberger_religionsfriede--d2e45471}}Pána Ježíše Krista\footnotetextA[1]
		{\label{edoc_ed000227_fg_kuttenberger_religionsfriede--d2e45471}Talmb., S. 514, Z. 10: pána našeho Jezu Krista; Fürstenb., S. 420, Z. 21: pána Ježíše.} dotýče, lidem obyčej majicím: buď pod obojí  neb pod jednú zpuosobú\footnotetextA
		{Ebd., S. 420, Z. 22: \textit{danach folgend} přijímati.}, o kterýž artikul \textsuperscript{\aalph{edoc_ed000227_fg_kuttenberger_religionsfriede--d2e45542}}útiskové  a mnohé nepřízně\footnotetextA[1]
		{\label{edoc_ed000227_fg_kuttenberger_religionsfriede--d2e45542}Talmb., S. 514, Z. 11: mnohé nepřízně a útiskowé; Fürstenb., S. 420, Z. 22: utiskowánie mnohá a nepřiezně.} byly jsú. I hledíc\footnotetextA
		{Talmb., S. 514, Z. 12 und Fürstenb., S. 420, Z. 22f.: hledíce.}my na zvyklost starodávní, nic nového zamysliti nechtíc, zdá  se nám, aby o ten artikul tento\footnotetextA
		{Talmb., S. 514, Z. 13: ten.} zápis mezi obojí  stranú byl v artikulech\footnotetextA
		{Ebd., S. 514, Z. 13 und Fürstenb., S. 420, Z. 24: artikulích.} dole psaných, \textsuperscript{\aalph{edoc_ed000227_fg_kuttenberger_religionsfriede--d2e45681}}jakož dole psán  jest\footnotetextA[1]
		{\label{edoc_ed000227_fg_kuttenberger_religionsfriede--d2e45681}Fehlt ebd., S. 420, Z. 24.} těmito slovy: \textsuperscript{\aalph{edoc_ed000227_fg_kuttenberger_religionsfriede--d2e45791}}kompaktáta a smlúvy koncilium bazilejského, ty v své a na své míře stůjte a v své moci zuostaňte: jakž samy v sobě jsú etc.·\footnotetextA[1]
		{\label{edoc_ed000227_fg_kuttenberger_religionsfriede--d2e45791}Fehlt ebd., S. 420, Z. 24.}  \pend 
\pstart\vspace{0.25\baselineskip}  My, Jindřich starší\footnotetextA
		{Ebd., S. 420, Z. 25: \textit{danach folgend} a.}, Jindřich mladší, \textsuperscript{\aalph{edoc_ed000227_fg_kuttenberger_religionsfriede--d2e45847}}jiným  jménem\footnotetextA[1]
		{\label{edoc_ed000227_fg_kuttenberger_religionsfriede--d2e45847}Ebd., S. 420, Z. 25: jináč.} Hynek,\footnotetextA
		{Ebd., S. 420, Z. 25: \textit{danach folgend} z božie milosti.} knížata minstrberská\footnotetextA
		{Korrigiert aus: Mistrberská.},\footnotetextA
		{Ebd., S. 420, Z. 26: \textit{danach folgend} hrabě Kladští əc.} Vok z Rožmberka etc., \textsuperscript{\aalph{edoc_ed000227_fg_kuttenberger_religionsfriede--d2e45919}}páni pořád\footnotetextA[1]
		{\label{edoc_ed000227_fg_kuttenberger_religionsfriede--d2e45919}Ebd., S. 420, Z. 26: a páni pořád əc.}, rytířstvo pořád, města pořád\footnotetextA
		{Ebd., S. 420, Z. 26: \textit{danach folgend} əc.}  A. B. C. D.\footnotetextA
		{Talmb., S. 514, Z. 17: A. B. C.; fehlt Fürstenb., S. 420, Z. 26.} etc., vyznáváme\footnotetextA
		{Ebd., S. 420, Z. 27: \textit{danach folgend} tiemto zápisem əc. əc.}, že hledíc\footnotetextA
		{Talmb., S. 514, Z. 18 und Fürstenb., S. 420, Z. 27: hledíce.} napřed  k chvále Pána Krista\footnotetextA
		{Talmb., S. 514, Z. 18 und Fürstenb., S. 420, Z. 27: boha.} všemohúcího, potom k jednotě a \textsuperscript{\aalph{edoc_ed000227_fg_kuttenberger_religionsfriede--d2e46061}}ku pokojení k\footnotetextA[1]
		{\label{edoc_ed000227_fg_kuttenberger_religionsfriede--d2e46061}Talmb., S. 514, Z. 19: pokojně; Fürstenb., S. 420, Z. 28: pokojné.} svornosti království tohoto:  majíc\footnotetextA
		{Talmb., S. 514, Z. 19: majíce.} srovnání mezi sebú skrze krále \textsuperscript{\aalph{edoc_ed000227_fg_kuttenberger_religionsfriede--d2e46113}}Jeho Milost\footnotetextA[1]
		{\label{edoc_ed000227_fg_kuttenberger_religionsfriede--d2e46113}Ebd., S. 514, Z. 20: \textit{aufgrund fehlerhafter Transkription} JM\textsuperscript{t} \textit{statt} geho milost; fehlt Fürstenb., S. 420, Z. 28.},  pána našeho milostivého, o ten artikul, kterýž mnohé časy mezi námi stál v ruoznici, kterýž\footnotetextA
		{Ebd., S. 420, Z. 29: což.} se přijímání Těla a Krve Pána\footnotetextA
		{Talmb., S. 514, Z. 21: \textit{danach folgend} Jezu.} Krista pod jednú zpuosobú neb pod\footnotetextA
		{Fehlt ebd., S. 514, Z. 22.} obojí· z víry naší každého dotýče, na·  \textsuperscript{\aalph{edoc_ed000227_fg_kuttenberger_religionsfriede--d2e46309}}Jeho královské Milosti\footnotetextA[1]
		{\label{edoc_ed000227_fg_kuttenberger_religionsfriede--d2e46309}Ebd., S. 514, Z. 22 und Fürstenb., S. 420, Z. 31: \textit{aufgrund fehlerhafter Transkription} JMK\textsuperscript{ti} \textit{statt} geho kralowſke miloſti.} jsme přestali. A takto se proti tomu\footnotetextA
		{Talmb., S. 514, Z. 22 und Fürstenb., S. 420, Z. 31: sobě.} zachovati máme bez pohnutí: i \textsuperscript{\aalph{edoc_ed000227_fg_kuttenberger_religionsfriede--d2e46404}}bez porušení všelikterakého i\footnotetextA[1]
		{\label{edoc_ed000227_fg_kuttenberger_religionsfriede--d2e46404}Talmb., S. 514, Z. 23: přerušení wšelikého a; Fürstenb., S. 420, Z. 32: přerušenie wšelijakého.} vymyšlených forteluov, počnúc\footnotetextA
		{Talmb., S. 514, Z. 23: počnúce.} od datum \textsuperscript{\aalph{edoc_ed000227_fg_kuttenberger_religionsfriede--d2e46458}}tohoto listu\footnotetextA[1]
		{\label{edoc_ed000227_fg_kuttenberger_religionsfriede--d2e46458}Ebd., S. 514, Z. 24 und Fürstenb., S. 420, Z. 32: listu tohoto.} za plných \textsuperscript{\aalph{edoc_ed000227_fg_kuttenberger_religionsfriede--d2e46488}}jedno a třidceti  \page{G3r}let\footnotetextA[1]
		{\label{edoc_ed000227_fg_kuttenberger_religionsfriede--d2e46488}Talmb., S. 514, Z. 24: XXXI let; Fürstenb., S. 420, Z. 32f.: xxxi let.} pořád zběhlých a přešlých\footnotetextA
		{Ebd., S. 420, Z. 33: přišlých.}, buď z nás kto živ,  nebo mrtev, pro žádnou\footnotetextA
		{Talmb., S. 514, Z. 25 und Fürstenb., S. 420, Z. 33: nižádnú.} věc, ani smrt \textsuperscript{\aalph{edoc_ed000227_fg_kuttenberger_religionsfriede--d2e46568}}nás kterého\footnotetextA[1]
		{\label{edoc_ed000227_fg_kuttenberger_religionsfriede--d2e46568}Talmb., S. 514, Z. 25 und Fürstenb., S. 420, Z. 34: kterého z nás.},  má rušen \textsuperscript{\aalph{edoc_ed000227_fg_kuttenberger_religionsfriede--d2e46598}}tento zápis býti\footnotetextA[1]
		{\label{edoc_ed000227_fg_kuttenberger_religionsfriede--d2e46598}Talmb., S. 514, Z. 25f.: býti tento zápis.}: ale v celosti \textsuperscript{\aalph{edoc_ed000227_fg_kuttenberger_religionsfriede--d2e46627}}takto zachován\footnotetextA[1]
		{\label{edoc_ed000227_fg_kuttenberger_religionsfriede--d2e46627}Ebd., S. 514, Z. 26: takto zachowáno; Fürstenb., S. 420, Z. 34: má býti zachován.}:  \pend 
\pstart\vspace{0.25\baselineskip}  Item najprvé každý z nás z\footnotetextA
		{Fehlt Talmb., S. 514, Z. 27 und Fürstenb., S. 420, Z. 35.} své víry bude· a  přijímání té Velebné svátosti moci užívati:  podlé dověření svého\footnotetextA
		{Talmb., S. 514, Z. 28: \textit{danach folgend} a.} svědomí.  \pend 
\pstart\vspace{0.25\baselineskip}  Item \textsuperscript{\aalph{edoc_ed000227_fg_kuttenberger_religionsfriede--d2e46744}}všickni kněží náši\footnotetextA[1]
		{\label{edoc_ed000227_fg_kuttenberger_religionsfriede--d2e46744}Fürstenb., S. 420, Z. 37: kněžie wšickni.}, kteréžkoli nyní máme neb potom za sebú míti budem\footnotetextA
		{Talmb., S. 514, Z. 30: budeme.}· na panství a zboží\footnotetextA
		{Ebd., S. 514, Z. 30 und Fürstenb., S. 420, Z. 38: zbožích.} našich dědických neb zástavných: máme \textsuperscript{\aalph{edoc_ed000227_fg_kuttenberger_religionsfriede--d2e46842}}míti k tomu\footnotetextA[1]
		{\label{edoc_ed000227_fg_kuttenberger_religionsfriede--d2e46842}Talmb., S. 514, Z. 30f.: k tomu jmíti.} a držeti, aby obyčeje\footnotetextA
		{Ebd., S. 514, Z. 31 und Fürstenb., S. 420, Z. 39: \textit{danach folgend} při.} přijímání Těla a  Krve Boží pod jednú zpuosobú neb pod obojí zpuosobú\footnotetextA
		{Fehlt ebd., S. 420, Z. 39.} slušně vésti mohli· a svobodně kážíc\footnotetextA
		{Talmb., S. 514, Z. 32: kážíce.} čtení  svaté a jiné spravedlivé Písmo\footnotetextA
		{Ebd., S. 514, Z. 33: \textit{danach folgend} swaté.} bez hanění\footnotetextA
		{Fürstenb., S. 420, Z. 40: \textit{danach folgend} a.}, kaceřovaní strany druhé: i všelijakých úštípkuov \textsuperscript{\aalph{edoc_ed000227_fg_kuttenberger_religionsfriede--d2e47032}}a podstrkuov\footnotetextA[1]
		{\label{edoc_ed000227_fg_kuttenberger_religionsfriede--d2e47032}Fehlt Talmb., S. 514, Z. 33.} podávajíc lidem obecným, jak  v těch farách zvyklost mají:\footnotetextA
		{Ebd., S. 514, Z. 34 und Fürstenb., S. 420, Z. 42: \textit{danach folgend} tajně.} ani zjevně žádného  nenutíce\footnotetextA
		{Talmb., S. 514, Z. 35: nenutíc.}. Neb byli-li by při které faře \textsuperscript{\aalph{edoc_ed000227_fg_kuttenberger_religionsfriede--d2e47131}}lidé, kteří\footnotetextA[1]
		{\label{edoc_ed000227_fg_kuttenberger_religionsfriede--d2e47131}Ebd., S. 514, Z. 35: kteří lidé; fehlt Fürstenb., S. 420, Z. 43.}  buď jeden, nebo \textsuperscript{\aalph{edoc_ed000227_fg_kuttenberger_religionsfriede--d2e47160}}dva, nebo víc\footnotetextA[1]
		{\label{edoc_ed000227_fg_kuttenberger_religionsfriede--d2e47160}Talmb., S. 514, Z. 35: wíce; Fürstenb., S. 420, Z. 43: wiec.} jináč\footnotetextA
		{Talmb., S. 514, Z. 36: jinak.}\footnotetextA
		{Ebd., S. 514, Z. 36 und Fürstenb., S. 420, Z. 43: \textit{danach folgend} při.} přijímání  obyčej majíc\footnotetextA
		{Talmb., S. 514, Z. 36: majíce.}: než\footnotetextA
		{Fürstenb., S. 420, Z. 43: \textit{danach folgend} ten.} kněz ten \textsuperscript{\aalph{edoc_ed000227_fg_kuttenberger_religionsfriede--d2e47234}}při podávání má\footnotetextA[1]
		{\label{edoc_ed000227_fg_kuttenberger_religionsfriede--d2e47234}Talmb., S. 514, Z. 36: má při podávání; Fürstenb., S. 420, Z. 43f.: má.} ve  Velebné svátosti, ten \textsuperscript{\aalph{edoc_ed000227_fg_kuttenberger_religionsfriede--d2e47272}}v přijímání své\footnotetextA[1]
		{\label{edoc_ed000227_fg_kuttenberger_religionsfriede--d2e47272}Talmb., S. 514, Z. 37: přijímáním swým.}\footnotetextA
		{Fürstenb., S. 420, Z. 44: swém.} muož \textsuperscript{\aalph{edoc_ed000227_fg_kuttenberger_religionsfriede--d2e47306}}svého spasení\footnotetextA[1]
		{\label{edoc_ed000227_fg_kuttenberger_religionsfriede--d2e47306}Talmb., S. 514, Z. 37: spasení swého.} hledati, kdež se jemu \textsuperscript{\aalph{edoc_ed000227_fg_kuttenberger_religionsfriede--d2e47328}}bude zdáti\footnotetextA[1]
		{\label{edoc_ed000227_fg_kuttenberger_religionsfriede--d2e47328}Ebd., S. 514, Z. 37-S. 515, Z. 1: zdáti bude.} podlé dověření spasení jeho: a\footnotetextA
		{Fehlt Fürstenb., S. 421, Z. 1.} mimo to v ničem jeho\footnotetextA
		{Ebd., S. 421, Z. 1: jich.} neobtěžuj, jako křesťanu\footnotetextA
		{Ebd., S. 421, Z. 2: křesťanuom.} přísluší.\pend 
\pstart\vspace{0.25\baselineskip}  \page{G3v}Item každý z nás poddané své\footnotetextA
		{Talmb., S. 515, Z. 3 und Fürstenb., S. 421, Z. 3: naše.}, buď na dědinách  neb na\footnotetextA
		{Fehlt Talmb., S. 515, Z. 3.} zástavách, \textsuperscript{\aalph{edoc_ed000227_fg_kuttenberger_religionsfriede--d2e47478}}kteréž nyní\footnotetextA[1]
		{\label{edoc_ed000227_fg_kuttenberger_religionsfriede--d2e47478}Fürstenb., S. 421, Z. 3f.: kteréžkoli.} máme a držíme\footnotetextA
		{Ebd., S. 421, Z. 4: \textit{danach folgend} a držeti.}, neb  míti\footnotetextA
		{Talmb., S. 515, Z. 4: jmíti a držeti.} budeme, v pokoji· při obyčeji přijímání  jich buď pod jednú neb pod obojí zpuosobú, v nichž  dověření k jich spasení jest\footnotetextA
		{Fürstenb., S. 421, Z. 5: (jest).}, zachovati\footnotetextA
		{Ebd., S. 421, Z. 5: \textit{danach folgend} je.} máme: žádnú \textsuperscript{\aalph{edoc_ed000227_fg_kuttenberger_religionsfriede--d2e47617}}mocí jich nenutíc\footnotetextA[1]
		{\label{edoc_ed000227_fg_kuttenberger_religionsfriede--d2e47617}Talmb., S. 515, Z. 6: jich mocí nenutíce; Fürstenb., S. 421, Z. 6: mocí jich nenutiec.} ani jich\footnotetextA
		{Fehlt ebd., S. 421, Z. 6.} připravujíc\footnotetextA
		{Talmb., S. 515, Z. 6: připravujíce.} \textsuperscript{\aalph{edoc_ed000227_fg_kuttenberger_religionsfriede--d2e47666}}skrze sami\footnotetextA[1]
		{\label{edoc_ed000227_fg_kuttenberger_religionsfriede--d2e47666}Ebd., S. 515, Z. 6: sami skrze; Fürstenb., S. 421, Z. 6: skrze se samy.} se· neb \textsuperscript{\aalph{edoc_ed000227_fg_kuttenberger_religionsfriede--d2e47689}}kněžstvo naše\footnotetextA[1]
		{\label{edoc_ed000227_fg_kuttenberger_religionsfriede--d2e47689}Ebd., S. 421, Z. 6: skrze kněžstvo.} bez jich vuole: aby\footnotetextA
		{Talmb., S. 515, Z. 7: \textit{danach folgend} každý.}  \textsuperscript{\aalph{edoc_ed000227_fg_kuttenberger_religionsfriede--d2e47725}}svého spasení hledal\footnotetextA[1]
		{\label{edoc_ed000227_fg_kuttenberger_religionsfriede--d2e47725}Fürstenb., S. 421, Z. 7: hleděl swého spasenie.} podlé dověření svého při  skutku té Velebné svátosti. A to má\footnotetextA
		{Talmb., S. 515, Z. 8 und Fürstenb., S. 421, Z. 8: \textit{danach folgend} buď.} v městech \textsuperscript{\aalph{edoc_ed000227_fg_kuttenberger_religionsfriede--d2e47792}}nebo  v\footnotetextA[1]
		{\label{edoc_ed000227_fg_kuttenberger_religionsfriede--d2e47792}Talmb., S. 515, Z. 9: w; Fürstenb., S. 421, Z. 8: neb.} městečkách neb ve vsech všudy zachováno\footnotetextA
		{Ebd., S. 421, Z. 8: \textit{danach folgend} od nás.} býti.  \pend 
\pstart\vspace{0.25\baselineskip}  Item žádných jiných kněží, než \textsuperscript{\aalph{edoc_ed000227_fg_kuttenberger_religionsfriede--d2e47876}}kteří nyní jsú  a kde a jak zastáni\footnotetextA[1]
		{\label{edoc_ed000227_fg_kuttenberger_religionsfriede--d2e47876}Ebd., S. 421, Z. 9: ty, kteříž nynie zastiženi.} jsú, \textsuperscript{\aalph{edoc_ed000227_fg_kuttenberger_religionsfriede--d2e47900}}jiný obyčej majíc\footnotetextA[1]
		{\label{edoc_ed000227_fg_kuttenberger_religionsfriede--d2e47900}Ebd., S. 421, Z. 9: majíc obyčej.} při podávání té Velebné svátosti, na ta místa \textsuperscript{\aalph{edoc_ed000227_fg_kuttenberger_religionsfriede--d2e47955}}dávati nemáme\footnotetextA[1]
		{\label{edoc_ed000227_fg_kuttenberger_religionsfriede--d2e47955}Ebd., S. 421, Z. 10: nemáme dáwati.}: než ty a takové, kteříž nyní zastáni jsú.  Pakli by který umřel, na to místo buď \textsuperscript{\aalph{edoc_ed000227_fg_kuttenberger_religionsfriede--d2e48017}}jiný také\footnotetextA[1]
		{\label{edoc_ed000227_fg_kuttenberger_religionsfriede--d2e48017}Talmb., S. 515, Z. 12f.: jiný taký; Fürstenb., S. 421, Z. 11: buď takowý.} /  dán· téhož obyčeje, jako první byl kněz.  Také jakož\footnotetextA
		{Ebd., S. 421, Z. 12: \textit{danach folgend} se jest.} před časy minulými: smlúva \textsuperscript{\aalph{edoc_ed000227_fg_kuttenberger_religionsfriede--d2e48097}}se jest\footnotetextA[1]
		{\label{edoc_ed000227_fg_kuttenberger_religionsfriede--d2e48097}Fehlt ebd., S. 421, Z. 12.}  \textsuperscript{\aalph{edoc_ed000227_fg_kuttenberger_religionsfriede--d2e48115}}a zavření stalo\footnotetextA[1]
		{\label{edoc_ed000227_fg_kuttenberger_religionsfriede--d2e48115}Talmb., S. 515, Z. 14: stala a zřízenie.} mezi námi obojími. A cedule \textsuperscript{\aalph{edoc_ed000227_fg_kuttenberger_religionsfriede--d2e48154}}toho jsú\footnotetextA[1]
		{\label{edoc_ed000227_fg_kuttenberger_religionsfriede--d2e48154}Ebd., S. 515, Z. 14: jsú toho; Fürstenb., S. 421, Z. 13: toho učiněny jsú.}. A čemu se jest ještě dosti nestalo, k tomu  my\footnotetextA
		{Fehlt Talmb., S. 515, Z. 15 und Fürstenb., S. 421, Z. 13.} se míti\footnotetextA
		{Talmb., S. 515, Z. 15: jmíti.} chceme, aby se tomu\footnotetextA
		{Ebd., S. 515, Z. 15 und Fürstenb., S. 421, Z. 14: \textit{danach folgend} dosti.} stalo mezi nynějš[ím]  časem a hodem svatého Václava \textsuperscript{\aalph{edoc_ed000227_fg_kuttenberger_religionsfriede--d2e48286}}najprv příští.\footnotetextA[1]
		{\label{edoc_ed000227_fg_kuttenberger_religionsfriede--d2e48286}Talmb., S. 515, Z. 16: najprwé příštího; Fürstenb., S. 421, Z. 14f.: nynie najprw příštieho.}  \pend 
\pstart\vspace{0.25\baselineskip}  \textgreek{ℂ} A což se Prahy dotýče, \textsuperscript{\aalph{edoc_ed000227_fg_kuttenberger_religionsfriede--spcrit9}}\textsuperscript{\aalph{edoc_ed000227_fg_kuttenberger_religionsfriede--spcrit11}}jakož král, Jeho Milost\textsuperscript{\aalph{edoc_ed000227_fg_kuttenberger_religionsfriede--spcrit11}}\begin{marginFoot}\footnotetextA[1]
		{\edlabel{edoc_ed000227_fg_kuttenberger_religionsfriede--spcrit11}\label{edoc_ed000227_fg_kuttenberger_religionsfriede--spcrit11}Fehlt Talmb., S. 515, Z. 17.}\end{marginFoot},  pán náš milostivý, s nimi smlúvu má\textsuperscript{\aalph{edoc_ed000227_fg_kuttenberger_religionsfriede--spcrit9}}\begin{marginFoot}\footnotetextA[1]
		{\edlabel{edoc_ed000227_fg_kuttenberger_religionsfriede--spcrit9}\label{edoc_ed000227_fg_kuttenberger_religionsfriede--spcrit9}Fehlt Fürstenb., S. 421, Z. 16.}\end{marginFoot}: přitom\footnotetextA
		{Talmb., S. 515, Z. 17: a při tom.}  my toho necháváme, aby to k\footnotetextA
		{Fehlt Fürstenb., S. 421, Z. 16.} srovnání těmi osoba-\page{G4r}mi, kteréž\footnotetextA
		{Ebd., S. 421, Z. 17: kteříž}to jednati mají\footnotetextA
		{Ebd., S. 421, Z. 17: \textit{danach folgend} k srownání.}, přišlo. Pakli by se to nestalo, to sobě vymieňujem, abychom o to mluviti neb\footnotetextA
		{Talmb., S. 515, Z. 19: a.} státi mohli podlé potřebnosti: aby\footnotetextA
		{Ebd., S. 515, Z. 20: \textit{danach folgend} to.} k nápravě přišlo, ač\footnotetextA
		{Ebd., S. 515, Z. 20: \textit{aufgrund fehlerhafter Transkription} až \textit{statt} acž.} se nám zdáti bude.  Pakli by kto \textsuperscript{\aalph{edoc_ed000227_fg_kuttenberger_religionsfriede--d2e48603}}mimo srovnání toto\footnotetextA[1]
		{\label{edoc_ed000227_fg_kuttenberger_religionsfriede--d2e48603}Fehlt Fürstenb., S. 421, Z. 19.} \textsuperscript{\aalph{edoc_ed000227_fg_kuttenberger_religionsfriede--d2e48615}}a zjednání\footnotetextA[1]
		{\label{edoc_ed000227_fg_kuttenberger_religionsfriede--d2e48615}Ebd., S. 421, Z. 19: jednánie.} králem \textsuperscript{\aalph{edoc_ed000227_fg_kuttenberger_religionsfriede--d2e48642}}Jeho Milostí\footnotetextA[1]
		{\label{edoc_ed000227_fg_kuttenberger_religionsfriede--d2e48642}Talmb., S. 515, Z. 21 und Fürstenb., S. 421, Z. 19: \textit{aufgrund fehlerhafter Transkription} JM\textsuperscript{tí} a \textit{statt} geho miłoſtij a.} [a]\footnotetextA
		{Ergänzung von Tomáš Havelka.} námi, z nás se vytrhna, proti to[m]u  co činil\footnotetextA
		{Talmb., S. 515, Z. 21 und Fürstenb., S. 421, Z. 20: učinil.}, buď z knížat, pánuo\footnotetextA
		{Talmb., S. 515, Z. 22: pánuow; Fürstenb., S. 421, Z. 20: z pánuow, z.}, rytířstva, neb z\footnotetextA
		{Fehlt Talmb., S. 515, Z. 22.} měst,  neb z knězstva: a naň\footnotetextA
		{Fürstenb., S. 421, Z. 20: \textit{danach folgend} to.} vyznáno \textsuperscript{\aalph{edoc_ed000227_fg_kuttenberger_religionsfriede--d2e48782}}a shledáno bylo\footnotetextA[1]
		{\label{edoc_ed000227_fg_kuttenberger_religionsfriede--d2e48782}Talmb., S. 515, Z. 22: bylo a shledáno; Fürstenb., S. 421, Z. 21: neb shledáno bylo.},  \textsuperscript{\aalph{edoc_ed000227_fg_kuttenberger_religionsfriede--spcrit13}}proto \textsuperscript{\aalph{edoc_ed000227_fg_kuttenberger_religionsfriede--spcrit15}}všickni my\textsuperscript{\aalph{edoc_ed000227_fg_kuttenberger_religionsfriede--spcrit15}}\begin{marginFoot}\footnotetextA[1]
		{\edlabel{edoc_ed000227_fg_kuttenberger_religionsfriede--spcrit15}\label{edoc_ed000227_fg_kuttenberger_religionsfriede--spcrit15}Talmb., S. 515, Z. 23: my wšickni.}\end{marginFoot}podlé povinnosti zavázku \textsuperscript{\aalph{edoc_ed000227_fg_kuttenberger_religionsfriede--spcrit17}}našeho  tohoto\textsuperscript{\aalph{edoc_ed000227_fg_kuttenberger_religionsfriede--spcrit17}}\begin{marginFoot}\footnotetextA[1]
		{\edlabel{edoc_ed000227_fg_kuttenberger_religionsfriede--spcrit17}\label{edoc_ed000227_fg_kuttenberger_religionsfriede--spcrit17}Talmb., S. 515, Z. 23: tohoto našeho.}\end{marginFoot} ze cti a\footnotetextA
		{Talmb., S. 515, Z. 23: \textit{danach folgend} z.} víry pro\footnotetextA
		{Ebd., S. 515, Z. 23: po.} jednoho upomínáni a haněni a dotýkáni býti\footnotetextA
		{Ebd., S. 515, Z. 24: \textit{danach folgend} nemáme.}, než ten\footnotetextA
		{Ebd., S. 515, Z. 24: \textit{danach folgend} neb ti.}, ktož by to \textsuperscript{\aalph{edoc_ed000227_fg_kuttenberger_religionsfriede--spcrit19}}přičinil aneb  přičinili\textsuperscript{\aalph{edoc_ed000227_fg_kuttenberger_religionsfriede--spcrit19}}\begin{marginFoot}\footnotetextA[1]
		{\edlabel{edoc_ed000227_fg_kuttenberger_religionsfriede--spcrit19}\label{edoc_ed000227_fg_kuttenberger_religionsfriede--spcrit19}Talmb., S. 515, Z. 24f.: přečinil neb přečinili.}\end{marginFoot}\textsuperscript{\aalph{edoc_ed000227_fg_kuttenberger_religionsfriede--spcrit13}}\begin{marginFoot}\footnotetextA[1]
		{\edlabel{edoc_ed000227_fg_kuttenberger_religionsfriede--spcrit13}\label{edoc_ed000227_fg_kuttenberger_religionsfriede--spcrit13}Fehlt Fürstenb., S. 421, Z. 21.}\end{marginFoot}. \textsuperscript{\aalph{edoc_ed000227_fg_kuttenberger_religionsfriede--d2e48961}}Než my všickni\footnotetextA[1]
		{\label{edoc_ed000227_fg_kuttenberger_religionsfriede--d2e48961}Fürstenb., S. 421, Z. 21: tehdy my wšichni.} slibujem \textsuperscript{\aalph{edoc_ed000227_fg_kuttenberger_religionsfriede--d2e48991}}slovem dole psaným\footnotetextA[1]
		{\label{edoc_ed000227_fg_kuttenberger_religionsfriede--d2e48991}Talmb., S. 515, Z. 25: dolepsaným slowem.} a závazkem: proti tomu, ktož by to\footnotetextA
		{Ebd., S. 515, Z. 26: co.} učinil  \textsuperscript{\aalph{edoc_ed000227_fg_kuttenberger_religionsfriede--d2e49023}}neb učinili\footnotetextA[1]
		{\label{edoc_ed000227_fg_kuttenberger_religionsfriede--d2e49023}Fehlt Fürstenb., S. 421, Z. 22.}, napřed králi \textsuperscript{\aalph{edoc_ed000227_fg_kuttenberger_religionsfriede--d2e49046}}Jeho Milosti\footnotetextA[1]
		{\label{edoc_ed000227_fg_kuttenberger_religionsfriede--d2e49046}Talmb., S. 515, Z. 26 und Fürstenb., S. 421, Z. 22: \textit{aufgrund fehlerhafter Transkription} JM\textsuperscript{ti} \textit{statt} geho miłoſti.}, pánu našemu milostivému\footnotetextA
		{Talmb., S. 515, Z. 27: \textit{danach folgend} nynějšímu.}, i \textsuperscript{\aalph{edoc_ed000227_fg_kuttenberger_religionsfriede--d2e49109}}králi Jeho Milosti potomkom\footnotetextA[1]
		{\label{edoc_ed000227_fg_kuttenberger_religionsfriede--d2e49109}Ebd., S. 515, Z. 27: \textit{aufgrund fehlerhafter Transkription} krále JM\textsuperscript{ti} potomkuom \textit{statt} krále geho miłoſti potomkuom; Fürstenb., S. 421, Z. 22: potomkuom.},  králom budúcím, ač by \textsuperscript{\aalph{edoc_ed000227_fg_kuttenberger_religionsfriede--d2e49156}}Jeho královské Milosti\footnotetextA[1]
		{\label{edoc_ed000227_fg_kuttenberger_religionsfriede--d2e49156}Talmb., S. 515, Z. 27 und Fürstenb., S. 421, Z. 23: JKM\textsuperscript{ti}.} \textsuperscript{\aalph{edoc_ed000227_fg_kuttenberger_religionsfriede--d2e49198}}z dopuštění Božího v tom a do toho času\footnotetextA[1]
		{\label{edoc_ed000227_fg_kuttenberger_religionsfriede--d2e49198}Ebd., S. 421, Z. 23: w tom z božieho dopuštěnie.} s\footnotetextA
		{Ebd., S. 421, Z. 23: \textit{danach folgend} tohoto.} světa sšel:  jehož, Pane Bože, ostřež. A který\footnotetextA
		{Talmb., S. 515, Z. 28 und Fürstenb., S. 421, Z. 24: kterýby.} král \textsuperscript{\aalph{edoc_ed000227_fg_kuttenberger_religionsfriede--spcrit30}}\textsuperscript{\aalph{edoc_ed000227_fg_kuttenberger_religionsfriede--spcrit35}}Jeho Milost\textsuperscript{\aalph{edoc_ed000227_fg_kuttenberger_religionsfriede--spcrit35}}\begin{marginFoot}\footnotetextA[1]
		{\edlabel{edoc_ed000227_fg_kuttenberger_religionsfriede--spcrit35}\label{edoc_ed000227_fg_kuttenberger_religionsfriede--spcrit35}Talmb., S. 515, Z. 28: \textit{aufgrund fehlerhafter Transkription} JM\textsuperscript{t} \textit{statt} geho miłoſt.}\end{marginFoot} neb králi po \textsuperscript{\aalph{edoc_ed000227_fg_kuttenberger_religionsfriede--spcrit33}}Jeho Milosti\textsuperscript{\aalph{edoc_ed000227_fg_kuttenberger_religionsfriede--spcrit33}}\begin{marginFoot}\footnotetextA[1]
		{\edlabel{edoc_ed000227_fg_kuttenberger_religionsfriede--spcrit33}\label{edoc_ed000227_fg_kuttenberger_religionsfriede--spcrit33}Talmb., S. 515, Z. 29: \textit{aufgrund fehlerhafter Transkription} JM\textsuperscript{ti} \textit{statt} geho miłoſti.}\end{marginFoot} byli i stráně druhé\textsuperscript{\aalph{edoc_ed000227_fg_kuttenberger_religionsfriede--spcrit30}}\begin{marginFoot}\footnotetextA[1]
		{\edlabel{edoc_ed000227_fg_kuttenberger_religionsfriede--spcrit30}\label{edoc_ed000227_fg_kuttenberger_religionsfriede--spcrit30}Fürstenb., S. 421, Z. 24: JM\textsuperscript{t} po JM\textsuperscript{ti}.}\end{marginFoot} takového\footnotetextA
		{Talmb., S. 515, Z. 29: takého.} v moc k opravě\footnotetextA
		{Fürstenb., S. 421, Z. 24: náprawě.} přivesti \textsuperscript{\aalph{edoc_ed000227_fg_kuttenberger_religionsfriede--spcrit21}}s Jich královskú\footnotetextA
		{Talmb., S. 515, Z. 29f.: Králowských.}  Milostí pomoci\textsuperscript{\aalph{edoc_ed000227_fg_kuttenberger_religionsfriede--spcrit21}}\begin{marginFoot}\footnotetextA[1]
		{\edlabel{edoc_ed000227_fg_kuttenberger_religionsfriede--spcrit21}\label{edoc_ed000227_fg_kuttenberger_religionsfriede--spcrit21}Fürstenb., S. 421, Z. 25: chcme s J\textsuperscript{ich}KM\textsuperscript{ti} pomoci a také.}\end{marginFoot} i strany druhé\footnotetextA
		{Ebd., S. 515, Z. 30: \textit{danach folgend} máme.}, aby ten to\footnotetextA
		{Fürstenb., S. 421, Z. 25: to ten.} skutečně napravil i kazán byl, \textsuperscript{\aalph{edoc_ed000227_fg_kuttenberger_religionsfriede--d2e49475}}jakžkoli\footnotetextA[1]
		{\label{edoc_ed000227_fg_kuttenberger_religionsfriede--d2e49475}Ebd., S. 421, Z. 26: jakžby se.} králi \textsuperscript{\aalph{edoc_ed000227_fg_kuttenberger_religionsfriede--d2e49491}}Jeho Milosti\footnotetextA[1]
		{\label{edoc_ed000227_fg_kuttenberger_religionsfriede--d2e49491}Talmb., S. 515, Z. 31 und Fürstenb., S. 421, Z. 26: \textit{aufgrund fehlerhafter Transkription} JM\textsuperscript{ti} \textit{statt} geho miłoſti.},  pánu naše[m]u milostivé[m]u, a radě z\footnotetextA
		{Fehlt Talmb., S. 515, Z. 31 und Fürstenb., S. 421, Z. 26.} stran obojích \textsuperscript{\aalph{edoc_ed000227_fg_kuttenberger_religionsfriede--spcrit23}}při  \textsuperscript{\aalph{edoc_ed000227_fg_kuttenberger_religionsfriede--spcrit25}}Jeho Milosti\textsuperscript{\aalph{edoc_ed000227_fg_kuttenberger_religionsfriede--spcrit25}}\begin{marginFoot}\footnotetextA[1]
		{\edlabel{edoc_ed000227_fg_kuttenberger_religionsfriede--spcrit25}\label{edoc_ed000227_fg_kuttenberger_religionsfriede--spcrit25}Talmb., S. 515, Z. 31: \textit{aufgrund fehlerhafter Transkription} JMti \textit{statt} geho miłoſti.}\end{marginFoot}\textsuperscript{\aalph{edoc_ed000227_fg_kuttenberger_religionsfriede--spcrit23}}\begin{marginFoot}\footnotetextA[1]
		{\edlabel{edoc_ed000227_fg_kuttenberger_religionsfriede--spcrit23}\label{edoc_ed000227_fg_kuttenberger_religionsfriede--spcrit23}Fehlt Fürstenb., S. 421, Z. 26.}\end{marginFoot} \textsuperscript{\aalph{edoc_ed000227_fg_kuttenberger_religionsfriede--d2e49603}}se zdáti bude za spravedlivé\footnotetextA[1]
		{\label{edoc_ed000227_fg_kuttenberger_religionsfriede--d2e49603}Talmb., S. 515, Z. 32 und Fürstenb., S. 421, Z. 26f.: zdáti se za sprawedliwé bude.}, aby \textsuperscript{\aalph{edoc_ed000227_fg_kuttenberger_religionsfriede--d2e49639}}tak  v moci a v té míře zase\footnotetextA[1]
		{\label{edoc_ed000227_fg_kuttenberger_religionsfriede--d2e49639}Ebd., S. 421, Z. 27: to tak w té moci a mieře zasě.} zuostalo a stálo, Jakož nahoře \textsuperscript{\aalph{edoc_ed000227_fg_kuttenberger_religionsfriede--d2e49673}}smluveno a\footnotetextA[1]
		{\label{edoc_ed000227_fg_kuttenberger_religionsfriede--d2e49673}Fehlt ebd., S. 421, Z. 27.} srovnáno\footnotetextA
		{Talmb., S. 515, Z. 33 und Fürstenb., S. 421, Z. 28: \textit{danach folgend} jest.}.\pend 
\pstart\vspace{0.25\baselineskip}  \page{G4v}\textsuperscript{\aalph{edoc_ed000227_fg_kuttenberger_religionsfriede--d2e49714}}Item k tomuto\footnotetextA[1]
		{\label{edoc_ed000227_fg_kuttenberger_religionsfriede--d2e49714}Ebd., S. 421, Z. 29: A k tomu.} zápisu my všickni\footnotetextA
		{Ebd., S. 421, Z. 29: wšichni.}, knížata a\footnotetextA
		{Fehlt Talmb., S. 515, Z. 34 und Fürstenb., S. 421, Z. 29.} páni, rytířstvo i\footnotetextA
		{Ebd., S. 421, Z. 29: a.} města, kterých\footnotetextA
		{Talmb., S. 515, Z. 34: kteréž.} jména napsána\footnotetextA
		{Fürstenb., S. 421, Z. 30: wepsána.}  jsú s\footnotetextA
		{Fehlt Talmb., S. 515, Z. 35.} strany naší, obyčej majíce přijímati Tělo a Krev \textsuperscript{\aalph{edoc_ed000227_fg_kuttenberger_religionsfriede--d2e49869}}Pána Ježíše Krista\footnotetextA[1]
		{\label{edoc_ed000227_fg_kuttenberger_religionsfriede--d2e49869}Fürstenb., S. 421, Z. 30: božie.} pod jednú zpuosobú,  pečeti své přivěsiti máme, \textsuperscript{\aalph{edoc_ed000227_fg_kuttenberger_religionsfriede--d2e49903}}a jižto\footnotetextA[1]
		{\label{edoc_ed000227_fg_kuttenberger_religionsfriede--d2e49903}Talmb., S. 515, Z. 36: a jichžto; Fürstenb., S. 421, Z. 31: kterýchžto také.} jména nejsú  \textsuperscript{\aalph{edoc_ed000227_fg_kuttenberger_religionsfriede--d2e49931}}vepsána v tento zápis\footnotetextA[1]
		{\label{edoc_ed000227_fg_kuttenberger_religionsfriede--d2e49931}Ebd., S. 421, Z. 31: napsána w tomto zápisu.}, ti své přiznávající\footnotetextA
		{Talmb., S. 515, Z. 37: přiznáwací; Fürstenb., S. 421, Z. 32: přiznawací.} listy  položiti mají při našem zápisu straně druhé pod obojí zpuosobu Tělo a Krev Pána Ježíše Krista\footnotetextA
		{Talmb., S. 515, Z. 38: \textit{danach folgend} přijímající; Fürstenb., S. 421, Z. 33: \textit{danach folgend} přijímajícím.}, \textsuperscript{\aalph{edoc_ed000227_fg_kuttenberger_religionsfriede--d2e50065}}kterýž  sme jim v moc dali a položili\footnotetextA[1]
		{\label{edoc_ed000227_fg_kuttenberger_religionsfriede--d2e50065}Ebd., S. 421, Z. 33: w moc jich dáti a položiti.}: a oni nám zase\footnotetextA
		{Ebd., S. 421, Z. 33: \textit{danach folgend} též.} takovýž\footnotetextA
		{Ebd., S. 421, Z. 34: \textit{danach folgend} učiniti mají.}.  \pend 
\pstart\vspace{0.25\baselineskip}  Pakli by kto \textsuperscript{\aalph{edoc_ed000227_fg_kuttenberger_religionsfriede--d2e50136}}k listu neb buď pečeti neb přiznávajícího listu nepoložili\footnotetextA[1]
		{\label{edoc_ed000227_fg_kuttenberger_religionsfriede--d2e50136}Talmb., S. 516, Z. 1f.: pečeti k listu neb přiznáwacieho listu nepoložil; Fürstenb., S. 421, Z. 34: buď pečeti k listu, neb přiznawacieho listu nepoložili.}, buď z\footnotetextA
		{Fehlt Talmb., S. 516, Z. 2 und Fürstenb., S. 421, Z. 34.} strany naší, neb druhé, slibujem napřed králi \textsuperscript{\aalph{edoc_ed000227_fg_kuttenberger_religionsfriede--d2e50204}}Jeho Milosti\footnotetextA[1]
		{\label{edoc_ed000227_fg_kuttenberger_religionsfriede--d2e50204}Talmb., S. 516, Z. 3 und Fürstenb., S. 421, Z. 35: \textit{aufgrund fehlerhafter Transkription} JM\textsuperscript{ti} \textit{statt} geho miłoſti.}, i sami  sobě, i straně druhé na toho každého, ktož by v tom  vuole své požívati chtěl a toho odpíral učiniti\footnotetextA
		{Fehlt ebd., S. 421, Z. 36.}, \textsuperscript{\aalph{edoc_ed000227_fg_kuttenberger_religionsfriede--d2e50310}}sobě pomoci\footnotetextA[1]
		{\label{edoc_ed000227_fg_kuttenberger_religionsfriede--d2e50310}Talmb., S. 516, Z. 4: sobě pomoc; Fürstenb., S. 421, Z. 36: pomoci sobě.} hrdly i statky podlé všeho přemožení  našeho, jako proti tomu rušiteli dobrého, svornosti a pokoje. A jeho žádný z nás nemá zastávati  ani fedrovati, než jeho \textsuperscript{\aalph{edoc_ed000227_fg_kuttenberger_religionsfriede--d2e50402}}zbaviti se\footnotetextA[1]
		{\label{edoc_ed000227_fg_kuttenberger_religionsfriede--d2e50402}Talmb., S. 516, Z. 6: zbaviti; Fürstenb., S. 421, Z. 38: se zbawiti.} všemi obyčeji  jako zprotivilého pánu svému a\footnotetextA
		{Ebd., S. 421, Z. 39: i.} vší obci. A to všickni\footnotetextA
		{Talmb., S. 516, Z. 7 und Fürstenb., S. 421, Z. 39: wšecko.} slíbili sme sami\footnotetextA
		{Fehlt Talmb., S. 516, Z. 7.}, každý sám za se i za naše herby\footnotetextA
		{Ebd., S. 516, Z. 8 und Fürstenb., S. 421, Z. 40: erby.} \textsuperscript{\aalph{edoc_ed000227_fg_kuttenberger_religionsfriede--d2e50540}}i potomky\footnotetextA[1]
		{\label{edoc_ed000227_fg_kuttenberger_religionsfriede--d2e50540}Ebd., S. 421, Z. 40: a potomníky.}, svú dobrú víru\footnotetextA
		{Ebd., S. 421, Z. 40: dobrú čistú.} křesťanskú, ctí a  věrú králi \textsuperscript{\aalph{edoc_ed000227_fg_kuttenberger_religionsfriede--d2e50585}}Jeho Milosti\footnotetextA[1]
		{\label{edoc_ed000227_fg_kuttenberger_religionsfriede--d2e50585}Talmb., S. 516, Z. 8 und Fürstenb., S. 421, Z. 40: \textit{aufgrund fehlerhafter Transkription} JM\textsuperscript{ti} \textit{statt} geho miłoſti.}, straně druhé i sami sobě  zdržeti\footnotetextA
		{Talmb., S. 516, Z. 9 und Fürstenb., S. 421, Z. 41: \textit{danach folgend} a.} zachovati ve všem, což v tomto zápisu \textsuperscript{\aalph{edoc_ed000227_fg_kuttenberger_religionsfriede--d2e50679}}se oznamuje,\footnotetextA[1]
		{\label{edoc_ed000227_fg_kuttenberger_religionsfriede--d2e50679}Talmb., S. 516, Z. 9: oznamuje se.} vyslovuje\footnotetextA
		{Fürstenb., S. 421, Z. 41: oznamuje.}: a zavírá, beze všeho porušení\footnotetextA
		{Talmb., S. 516, Z. 10 und Fürstenb., S. 421, Z. 42: přerušenie.}, obmeškání a zanetbání.\pend 
\pstart\vspace{0.25\baselineskip}  \page{G5r}Pakli by který proti tomu učinil a nenapravil vedle rozkázání\footnotetextA
		{Talmb., S. 516, Z. 11 und Fürstenb., S. 421, Z. 43: \textit{danach folgend} a rozeznánie.} krále \textsuperscript{\aalph{edoc_ed000227_fg_kuttenberger_religionsfriede--d2e50807}}Jeho Milosti\footnotetextA[1]
		{\label{edoc_ed000227_fg_kuttenberger_religionsfriede--d2e50807}Talmb., S. 516, Z. 11 und Fürstenb., S. 421, Z. 43: \textit{aufgrund fehlerhafter Transkription} JM\textsuperscript{ti} \textit{statt} geho miłoſti.} a osob s obú stranou  vydaných, na toho \textsuperscript{\aalph{edoc_ed000227_fg_kuttenberger_religionsfriede--d2e50851}}vyznáváme a seznáváme\footnotetextA[1]
		{\label{edoc_ed000227_fg_kuttenberger_religionsfriede--d2e50851}Ebd., S. 421, Z. 44: wyznáme a seznáme.}, že jest  proti své cti učinil a ji ztratil, jako ten, [ježto]\footnotetextA
		{Ergänzt nach ebd., S. 421, Z. 44.} nemiluje  svornosti a jednoty, a \textsuperscript{\aalph{edoc_ed000227_fg_kuttenberger_religionsfriede--spcrit36}}svému slibu dosti neučiniv /  a proti tomu zápisu učiniti\footnotetextA
		{Talmb., S. 516, Z. 14: učiniw.}\textsuperscript{\aalph{edoc_ed000227_fg_kuttenberger_religionsfriede--spcrit36}}\begin{marginFoot}\footnotetextA[1]
		{\edlabel{edoc_ed000227_fg_kuttenberger_religionsfriede--spcrit36}\label{edoc_ed000227_fg_kuttenberger_religionsfriede--spcrit36}Fürstenb., S. 422, Z. 1: swým slibuom dosti neučinil.}\end{marginFoot}. \textsuperscript{\aalph{edoc_ed000227_fg_kuttenberger_religionsfriede--spcrit27}}Tomu na svědomí\footnotetextA
		{Ebd., S. 516, Z. 14: jistotu.}  pečeti naše. A my, nadepsaný král Vladislav,  při každém spisu kázali jsme pečet naši královskú  přitisknúti.\textsuperscript{\aalph{edoc_ed000227_fg_kuttenberger_religionsfriede--spcrit27}}\begin{marginFoot}\footnotetextA[1]
		{\edlabel{edoc_ed000227_fg_kuttenberger_religionsfriede--spcrit27}\label{edoc_ed000227_fg_kuttenberger_religionsfriede--spcrit27}Fürstenb., S. 422, Z. 1: K kterémužto zápisu na jistotu a pamět pečeti naše wlastnie dali sme etc.}\end{marginFoot}\pend 
\pstart\vspace{0.25\baselineskip}  \page{249r} Item \textsuperscript{\aalph{edoc_ed000227_fg_kuttenberger_religionsfriede--d2e51040}}na tomto\footnotetextA[1]
		{\label{edoc_ed000227_fg_kuttenberger_religionsfriede--d2e51040}Fürstenb., S. 422, Z. 3: na tom.} jest zuostáno a svoleno od kniežat, pá[nuo]v,  rytieřstva i měst obojie strany, že mají králi  Jeho [Milos]ti\footnotetextA
		{Ebd., S. 422, Z. 4: JMti.} slíbiti / tyto artikule dolepsané, A to těmito  slovy, kazdý osobně Jeho [Milos]ti\footnotetextA
		{Ebd., S. 422, Z. 5: JMti.} rukú dánim slíbiti má,  napřed kniežata, páni i rytieřstvo: |Takto slibují|\footnotetextA
		{Überschrift des folgenden Paragraphen, der im Fürstenberg Manuskript mit roter Tinte am Ende des vorhergehenden Paragraphen steht. Hier und im Folgenden durch senkrechte Striche gekennzeichnet.}  \textsuperscript{\aalph{edoc_ed000227_fg_kuttenberger_religionsfriede--d2e51234}}Vaší královské Mi[los]ti\footnotetextA[1]
		{\label{edoc_ed000227_fg_kuttenberger_religionsfriede--d2e51234}Ebd., S. 422, Z. 6: WKMti.} slibuji, jestliže by ktokoli[vě]k  proti právu zemské[m]u starodávniemu / chtěl co uči[ni]ti  a je rušiti, anebo-li je[m]u dosti učiniti nechtěl podlé  nálezuov panských, že toho věrně, ctně, pravě,  křesťansky, beze všie zlé lsti podlé možnosti své pomáhati chci, na takového každého jako na dobrého člověka  slušie / podlé práv a svobod našich starodávních,  aby se právu zemskému dosti stalo / a každý, kniežata,  pání, rytieřstvo, Pražané i jiná města při svých  práviech a svobodách aby zachováni byli, a Jeho [Milos]t\footnotetextA
		{Ebd., S. 422, Z. 22: JMKá.}  královská ráčí jich všech obránce býti, A my  podlé Jeho [Milos]ti\footnotetextA
		{Ebd., S. 422, Z. 13: JMti.} všickní. |dále slibují|  Item toto také slíbiti mají, že oboje\footnotetextA
		{Ebd., S. 422, Z. 14: obojí.} strana nemají sebe haněti ani utiskati, buďto pro vieru,  neb pro kterú jinú věc /  (než o vieru / aby hledě[n]o  bylo ku právu tomu, k kterémuž kto příleží, jakožto za  předkuov Jeho [Milos]ti\footnotetextA
		{Ebd., S. 422, Z. 16: JMti.} / bývalo), ani svým poddaným dopúštěti. /  Pakli by kto to přestúpil, na takového pod týmž závazkem /  též Jeho [Milos]ti\footnotetextA
		{Ebd., S. 422, Z. 18: JMti.} slibujem radní a pomocní býti, na kohož by  to v pravdě uvedeno bylo. |Takto města slibují|  Item / což se Pražan i jiných měst dotýče, takto  mají králi Jeho [Milos]ti\footnotetextA
		{Ebd., S. 422, Z. 20: JMti.} slíbiti a přiřéci těmito slovy:  purgmistr a konšelé měst pražs[ký]ch od sebe i od svých  obcí / a z jiných měst poslové\footnotetextA
		{ Fehlt ebd., S. 422, Z. 22.}, kteříž sú, též od sebe  i od svých obcí, jestliže by kto, jakož nahoře psá[n]o  \page{249v}stojí, proti právu zemské[m]u a nálezuom panským dosti  učiniti nechtěl a chtěl své vuole požívati, pod těmi  závazky a přísahami, jakož jsú prve Jeho [Milos]ti\footnotetextA
		{Ebd., S. 423, Z. 3: JMti.} zavázáni,  na takového každého věrně, ctně, pravě, křesťansky,  beze všie zle lsti pomáhati, jako na dobré lidi a poddané Jeho [Milos]ti\footnotetextA
		{Ebd., S. 423, Z. 4: JMti.} slušie, vedlé práv a svobod svých. / A též což  se haněnie nebo útiskuov dotýče, jestliže by kto proti  tomu činil, že mají na každého takového radní i pomocní býti, podlé své možnosti, jestliže by kto právu zemskému / neb městskému dosti uči[ni]ti nechtěl,  k kterémuž příleží. / |O zápisích takto|  Item zápisové všickní, kteřížkoli jsú společní, buďto  mezi pány, rytieřstvem i městy, ti aby před \textsuperscript{\aalph{edoc_ed000227_fg_kuttenberger_religionsfriede--d2e52590}}Jeho [Milos]tí  královskú\footnotetextA[1]
		{\label{edoc_ed000227_fg_kuttenberger_religionsfriede--d2e52590}Ebd., S. 423, Z. 10: JMKú.} položení byli, když koli Jeho [Milos]t k[rálo]v[ská]\footnotetextA
		{Ebd., S. 423, Z. 10: JMKá.} / se pány a  vladykami v súd zemský zasede, neb to vedle práva  má býti, že právo svobodné má býti / a viece jiných  žádných aby nebylo bez Jeho\footnotetextA
		{Ebd., S. 423, Z. 12: JMti.} vuole a vědomie. Než  což se té smlúvy dotýče, kterúž král Jeho [Milos]t\footnotetextA
		{Ebd., S. 423, Z. 13: JMt.} s Pražany  má, při té smlúvě Jeho [Milos]t je zuostaviti ráčí, tak  jakož sama v sobě jest. |o těch, keří by nebyli|  Item \textsuperscript{\aalph{edoc_ed000227_fg_kuttenberger_religionsfriede--d2e52865}}keříž by koli\footnotetextA[1]
		{\label{edoc_ed000227_fg_kuttenberger_religionsfriede--d2e52865}Ebd., S. 423, Z. 15: kteřížbykoli.} zde v Praze nebyli, buďto z pánuov  nebo z rytieřstva, ti aby konečně ve čty[ře]ch ned[ělí]ch po osazení súdu k králi Jeho [Milos]ti\footnotetextA
		{Ebd., S. 423, Z. 16: JMti.} přijeli / a k takov[ém]u slibu se přiznali / tak, jakož napřed položeno jest. Pakli by kto toho  neučinil / a kterak toho zanetbal, proti takovému \textsuperscript{\aalph{edoc_ed000227_fg_kuttenberger_religionsfriede--d2e53074}}Jeho [Milos]t  královská\footnotetextA[1]
		{\label{edoc_ed000227_fg_kuttenberger_religionsfriede--d2e53074}Ebd., S. 423, Z. 18: JMKá.} má se mieti / a jej k tomu připraviti, a my  všichní Jeho Mi[los]ti\footnotetextA
		{Ebd., S. 423, Z. 19: JMti.} v tom máme radní a pomocní  býti. \pend
\endnumbering
\end{document}
